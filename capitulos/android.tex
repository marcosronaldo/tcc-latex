\chapter[Sistema Operacional Android]{Sistema Operacional Android\footnotemark}
\label{cap:android-os}
\footnotetext{Este capítulo é baseado na documentação oficial disponível em \citeonline{googledev}}

Este capítulo descreve um pouco o sistema Android, e tem o intuito de ser uma documentação básica resumida, devido a falta de documentação em português sobre a plataforma, visto que a principal fonte das informações, a documentação oficial em uma página web, se encontra apenas em inglês. Dessa forma, algumas seções se apresentam relativamente extensas e boa parte das informações contidas em algumas delas pode ser encontrada no site oficial em inglês. 
Ainda neste capítulo também foram utilizados conceitos da engenharia de software para contextualizar etapas de desenvolvimento de software em plataformas móveis, e em especial, no Android.

\section{Descrição Geral}

O Android é um sistema operacional para dispositivos móveis com base em kernel linux modificado, com várias bibliotecas modificadas ou refeitas, de forma a deixar o sistema tão eficiente quanto possível para o hardware limitado que os dispositivos alvo apresentam. A exemplo disso está a biblioteca C \textit{Bionic}, que foi desenvolvida para menor consumo de espaço físico, memória e poder de processamento que as bibliotecas padrão C como a GNU C (glibc) \cite{devos2014}. Aplicações desenvolvidas para Android são feitas essencialmente em linguagem Java, com a possibilidade de utilizar outras linguagens como C e C++ através da \textit{Java native interface} (JNI). 

O sistema Android tira vantagem do kernel linux no que diz respeito a identificação e separação de processos rodando no sistema, atribuindo a cada aplicação um \textit{UID} (\textit{User Identification}), e executando cada uma em um processo diferente, isolando umas das outras.  Independentemente de a aplicação ser desenvolvida em Java ou com código nativo, essa separação de processos do kernel, conhecida como \textit{Application Sandbox}, garante que a aplicação está isolada das demais e portanto sujeita aos mesmos mecanismos de segurança inclusive que os aplicativos do sistema, como contatos, câmera, entre outros. 

Nas versões anteriores ao Lollipop, cada uma dessas aplicações no sistema funcionava em uma instância diferente da \textit{Dalvik Virtual Machine} (DVM), enquanto atualmente a DVM foi substituída pela Android Run Time (ART), introduzida opcionalmente desde a versão kitkat. Ambas são máquinas virtuais semelhantes a Java Virtual Machine (JVM). Códigos em Java são compilados e traduzidos para formato .dex (\textit{dalvik executable}), que é executado pela DVM, semelhante ao formato .jar do Java. Enquanto a DVM utiliza \textit{just-in-time compilation}, compilando trechos do código para execução nativa em tempo de execução, a nova ART introduz o \textit{Ahead-of-time compilation}, realizando compilações em tempo de instalação. Embora a instalação possa levar mais tempo dessa forma, essa mudança permite que os aplicativos tenham maior performance em sua execução. Esse isolamento de aplicativos, onde cada um é executado em sua própria instancia da máquina virtual, permite que uma falha em um processo de uma aplicação não tenha impacto algum em outra aplicação.  

Para interagir com determinados serviços do sistema bem como outras aplicações, uma aplicação deve ter os privilégios correspondentes a ação que deseja executar. Por exemplo, o desenvolvedor pode solicitar ao sistema que sua aplicação tenha acesso a internet, privilégio não concedido por padrão pelo sistema. O usuário então no momento da instalação dessa aplicação é informado que a mesma deseja acesso a internet, e ele deve permitir acesso se quiser concluir a instalação. Todas as permissões requisitadas pelo desenvolvedor e necessárias para o aplicativo realizar suas funções são listadas no momento de instalação, e todas devem ser aceitas, caso contrário a instalação é cancelada. Não é permitido ao usuário selecionar quais permissões ele quer conceder e quais rejeitar à aplicação sendo instalada, tendo apenas as opções de aceitar todas elas, ou rejeitar a instalação.

O Android procura ser o mais seguro e facilmente utilizado sistema móvel, modificando a forma que várias tarefas são executadas para alcançar esse objetivo, como por exemplo fazer o isolamento de aplicações utilizando a separação de processos e usuários do kernel linux para gerenciar aplicativos instalados e proteger os dados dos mesmos. Aplicativos devem ser assinados e obrigatoriamente isolados uns dos outros (incluindo aplicativos do sistema) e devem possuir permissões explícitas para acessar recursos do sistema e outros aplicativos. As decisões arquiteturais relacionadas a segurança foram tomadas desde o início do ciclo de desenvolvimento do sistema, e continuam sendo prioridade.

Todo o código do sistema, incluindo a bionic, a Dalvik Virtual Machine e Android Run Time, é aberto para contribuição de qualquer desenvolvedor. Através do \textit{Android Open Sorce Project} (AOSP), as fabricantes de dispositivos obtém o código do sistema, modificam conforme desejarem, adicionam aplicativos próprios e distribuem com seus produtos.

\section{Estrutura de uma aplicação}

Aplicações no Android são construídas a partir de quatro tipos de componentes principais: \textit{Activities, Services, Broadcast Receivers}, e \textit{Content Providers} \cite{heuser2014}. 

\begin{enumerate}
\item Uma \textit{Activity} é basicamente o código para uma tarefa bem específica a ser realizada pelo usuário, e apresenta uma interface gráfica(\textit{Graphic User Interface}) para a realização dessa tarefa.

\item \textit{Services} são tarefas que são executadas em background, sem interação com o usuário. \textit{Services} podem funcionar no processo principal de uma aplicação ou no seu próprio processo. Um bom exemplo de \textit{services} são os tocadores de músicas. Mesmo que sua interface gráfica não esteja mais visível, é esperado que a música continue a tocar, mesmo se o usuário estiver interagindo com outro aplicativo. 

\item \textit{Broadcast Receiver} é um componente que é chamado quando um \textit{Intent} é criado e enviado via broadcast por alguma aplicação ou pelo sistema. \textit{Intents} são mecanismos para comunicação entre processos, podendo informar algum evento, ou transmitir dados de um para o outro.
Um aplicativo pode receber um \textit{Intent} criado por outro aplicativo, ou mesmo receber \textit{intents} do próprio sistema, como por exemplo informação de que a bateria está fraca ou de que uma busca por dispositivos \textit{bluetooth} previamente requisitada foi concluída. 

\item \textit{Content providers} são componentes que gerenciam o acesso a um conjunto de dados. São utilizados para criar um ponto de acesso a determinada informação para outras aplicações. Para os contatos do sistema, por exemplo, existe um \textit{Content Provider} responsável por gerenciar leitura e escrita desses contatos. 
\end{enumerate}


Cada um desses componentes pode funcionar independente dos demais. O sistema Android foi desenvolvido dessa forma para que uma tarefa mais complexa seja concluída com a ajuda e interação de vários desses componentes independente da aplicação a qual eles pertencem, não necessitando que um desenvolvedor crie mecanismos para todas as etapas de uma atividade mais longa do usuário.  

Para executar a tarefa de ler um email, por exemplo, um usuário instala um aplicativo de gerenciamento de emails. Então ele deseja abrir um anexo de um email que está em formato PDF. O aplicativo de email não precisa necessariamente prover um leitor de PDF para que o usuário consiga ter acesso a esse anexo. Ele pode mandar ao sistema a intenção de abrir um arquivo PDF a partir de um \textit{Intent}, e então o sistema encontra um outro componente que pode fazer isso, e o instancia. Caso mais de um seja encontrado, o sistema pergunta para o usuário qual é o componente que ele deseja utilizar. O sistema então invoca uma \textit{Activity} de um outro aplicativo para abrir esse arquivo. Continuando no mesmo exemplo, o usuário clica em um link dentro do arquivo pdf. Esse aplicativo, por sua vez, pode enviar ao sistema a intenção de abrir um endereço web, que mais uma vez encontra um aplicativo capaz de o fazer.  

É importante perceber que para uma atividade mais complexa de interação com o usuário, vários aplicativos são envolvidos sem que os mesmos tenham conhecimento dos demais. Cada componente se ``registra'' no sistema para realizar determinada tarefa, e o sistema se encarrega de encontrar os componentes adequados para cada situação. Esse registro dos componentes é realizado através do AndroidManifest.xml, que é um arquivo incluso em toda aplicação sendo instalada. Ele reúne todos os componentes de uma aplicação, as permissões necessárias para acessar cada um deles, e as permissões que eles utilizam, bem como outras informações. 

Uma vez que os componentes de uma aplicação podem ser utilizados por outras aplicações, é necessário um controle maior sobre quem pode ter acesso a cada um deles. Cada desenvolvedor pode criar permissões customizadas para seus componentes, e exigir que o aplicativo que requisite a tarefa tenha essa permissão para acessar o componente. Da mesma forma, o aplicativo que criou essa permissão determina os critérios para conceder a mesma para outros aplicativos. Um simples exemplo de uso desse mecanismo é o fato de uma empresa apenas criar vários aplicativos para tarefas distintas e querer integração entre os mesmos. O desenvolvedor pode definir uma permissão específica para acessar um dos seus \textit{Content Providers}, por exemplo, e definir que apenas aplicativos com a mesma assinatura (assinados pelo mesmo desenvolvedor) possam receber essa permissão. Dessa forma, todos os aplicativos desenvolvidos por essa empresa podem ter acesso aos dados gerenciados por esse \textit{Content Provider}, enquanto as demais aplicações não tem esse acesso. Aplicativos pré instalados ou internos do sistema podem conter um tipo de permissão específica que só é dada a aplicativos do sistema, e não pode ser obtida por nenhum outro aplicativo instalado pelo usuário.

\section{Diversidade e Compatibilidade}

O Android foi projetado para executar em uma imensa variedade de dispositivos, de telefones a \textit{tablets} e televisões. Isso é muito interessante no ponto de vista do desenvolvedor, que tem como mercado para seu software usuários de diversos dispositivos de diversas marcas diferentes. Entretanto, isso trás uma necessidade de fazer uma interface flexível, que permita que um aplicativo seja utilizável em vários tipos de dispositivos, com vários tamanhos de tela. Para facilitar esse problema, o Android oferece um \textit{framework} em que se pode prover recursos gráficos distintos e específicos para cada configuração de tela, publicando então um aplicativo apenas que se apresenta de forma diferente dependendo do dispositivo onde ele está sendo executado.  

A interface gráfica no Android é essencialmente construída em XML, e tem um padrão de navegação para as aplicações, embora fique a critério do desenvolvedor a aparência de sua aplicação. O desenvolvedor por criar, por exemplo, uma interface gráfica com arquivos XML para cada tamanho de tela, e também diferenciar entre modo paisagem e modo retrato. Entretanto, em se tratando de interface gráfica, vários componentes vão sendo adicionados a API Android ao longo de sua evolução, e portanto vários recursos gráficos necessitam de uma versão mínima do sistema para serem utilizados. Utilizar um recurso presente apenas a partir da versão ICS 4.0.4 (\textit{Ice Cream Sandwich}), por exemplo, implica que o aplicativo não tenha compatibilidade com versões anteriores do sistema.  

Da mesma forma, devido a diversa variedade de modelos e fabricantes de hardware, é preciso ficar atento aos recursos de hardware disponíveis para cada dispositivo. Alguns sensores hoje mais comuns aos novos dispositivos sendo lançados no mercado não existiam em modelos mais antigos. O desenvolvedor pode especificar no \textit{Android manifest} os recursos necessários para o funcionamento completo de sua aplicação, de forma que a mesma seja apenas instalada em dispositivos que os apresentarem. Também pode ser feita uma checagem em tempo de execução e apenas desativar uma funcionalidade do aplicativo caso algum recurso de hardware não esteja disponível no dispositivo, se isso for o desejo do desenvolvedor. De forma geral, é relativamente simples a forma com que o desenvolvedor especifica os dispositivos alvo para sua aplicação, tornando essa grande diversidade de dispositivos mais vantajosa do que dispendiosa. 

Android é um sistema livre, que pode ser utilizado em modificado e utilizado segundo a licença Apache versão 2.2. \citeonline{shanker2011} apresenta alguns tópicos relacionados a portabilidade do sistema Android em um novo hardware. Embora não seja discutida nesse documento, a relativamente fácil portabilidade do sistema para vários tipos de hardware foi uma das razões que levaram seu rápido crescimento, fazendo com que várias fabricantes possam fazer uso do mesmo sistema e lançar vários tipos de dispositivos distintos no mercado, com diferentes \textit{features} e preços, alcançando parcelas do mercado que possuem condições de aquisição muito variadas.

\section{Engenharia de Software aplicada ao Android}

Engenharia de software é definida pela \cite{swebok} como aplicação de uma abordagem sistemática, disciplinada e mensurável ao desenvolvimento, operação e manutenção de software. As subseções desta seção citam alguns dos processos da engenharia de software que são importantes em todo o ciclo de vida do produto de software, que inclui desde a concepção do produto até a manutenção do mesmo em ambiente de produção, no contexto de plataformas móveis, e em específico, no Android.

\subsection{Seleção de plataformas móveis}

\citeonline{platformchoice} Por exemplo, propõe 3 critérios para seleção de plataformas móveis no ponto de vista do desenvolvedor:
\begin{enumerate}
\item Remuneração - Relacionado ao número de potenciais clientes que podem adquirir o produto. Plataforma com grande número de usuários e crescimento constante é de grande valia para desenvolvedores. Um ponto centralizado de venda de aplicativos também é um atrativo para a comercialização dos softwares desenvolvidos.
\item Carreira - As oportunidades que o desenvolvimento para a plataforma pode gerar. A possibilidade de trabalhar para grandes e renomadas empresas no mercado pode ser um fator decisivo para a escolha da plataforma. Para aumentar sua credibilidade na comunidade de desenvolvedores e ganhar visibilidade no mercado, \citeonline{platformchoice} sugere que o desenvolvedor participe de projetos de software livre, o que é facilitado quando a própria plataforma móvel é aberta.
\item Liberdade - Liberdade criativa para desenvolver. O programador deve sentir que na plataforma ele pode programar o que quiser. Uma plataforma consolidada com excelentes kits de desenvolvimento e dispositivos atrativos do ponto de vista de hardware e sistema operacional atraem muitos desenvolvedores. Plataformas fechadas e com muitas restrições tendem a afastar desenvolvedores que querem essa liberdade, enquanto plataformas abertas apresentam maior liberdade para desenvolvimento.
\end{enumerate}

Com seu grande crescimento e consequente tomada da maior fatia do mercado de dispositivos móveis, a plataforma Android consegue ser bastante atrativa no ponto de vista primeiro tópico, em contraste com as demais plataformas do mercado, como WindowsPhone~\footnote{\url{http://www.windowsphone.com/}} e iOS~\footnote{\url{https://www.apple.com/br/ios/}}. Mais e mais empresas aparecem no contexto do Android tanto em desenvolvimento de hardware quanto software, e as oportunidades de trabalho crescem juntamente com o crescimento da própria plataforma, como sugerido no segundo tópico. Por ser aberto, a plataforma Android também permite que desenvolvedores enviem suas contribuições e correções de bugs ao próprio sistema operacional, aumentando sua visibilidade. Da mesma forma, o Android também apresenta um sistema totalmente aberto e com kits de desenvolvimento consolidados e extensiva documentação disponível online~\footnote{\url{http://developer.android.com/}}, no mínimo se equiparando ao seu principal concorrente iOS em liberdade de desenvolvimento na data de escrita desse documento.

Segundo \citeonline{eswissues}, alguns aspectos devem ser pensados quando desenvolvendo software para dispositivos móveis:

\begin{enumerate}
\item Requisitos de interação com outras aplicações;
\item Manipulação de sensores;
\item Requisitos web que resultam em aplicações hibridas (\textit{mobile} - \textit{web});
\item Diferentes famílias de hardware;
\item Requisitos de segurança contra aplicações mal intencionadas que comprometem o uso do sistema;
\item Interface de Usuário projetadas para funcionar com diversas formas de interação e seguir padrões de design da plataforma;
\item Teste de aplicações móveis são em geral mais desafiadores pois são realizados de maneira diferente da maneira tradicional;
\item Consumo de energia;
\end{enumerate}

Todos esses tópicos mencionados são muito importantes para o desenvolvimento em várias plataformas móveis, e modificam a forma com que várias atividades da engenharia de software devem ser abordadas. A plataforma Android tem sua arquitetura projetada para atender a vários desses quesitos:
\begin{itemize}
\item As aplicações são isoladas e se comunicam de forma unificada com outras aplicações ou com componentes e recursos do sistema;
\item A linguagem Java de programação da uma imensa liberdade de utilizar diversas bibliotecas e \textit{frameworks} desenvolvidos anteriormente para o Java;
\item A camada de máquina virtual correspondente a DVM e ART permite uma abstração do uso dos componentes físicos e aumenta assim a compatibilidade com diversos tipos e famílias de hardware;
\item A segurança foi prioridade desde os primeiros estágios de desenvolvimento, tentando prever inclusive ataques de engenharia social que tentam convencer o usuário a instalar aplicações maliciosas em seu dispositivo;
\item A interface de usuário dos aplicativos é extremamente customizável, ainda podendo manter facilmente um padrão de navegação;
\item Testes no sistema foram projetados para cada componente isoladamente, tentando facilitar o processo de teste de aplicativos;
\item Recursos do sistema tem controle minucioso para melhor gerenciamento de uso de energia.
\end{itemize}

Tomando os critérios apresentados como base, a escolha da plataforma Android para desenvolvimento neste trabalho é feita de forma clara.

\subsection{Requisitos}

Área de conhecimento em requisitos de software é responsável pela elicitação, análise, especificação e validação de requisitos de software, bem como a manutenção gerenciamento desses requisitos durante todo o ciclo de vida do produto \cite{swebok}.

Requisitos de software representam as necessidades de um produto, condições que ele deve cumprir para resolver um problema do mundo real que o software pretende atacar. Essas necessidades podem ser funcionalidades que o software deve apresentar para o usuário, chamados requisitos funcionais, ou outras condições que restringem as funcionalidades de alguma forma, seja por exemplo sobre tempo de execução, requisitos de segurança ou outras restrições, conhecidas como requisitos não funcionais.

Requisitos não funcionais são críticos para aplicações móveis, e estas podem precisar se adaptar dinamicamente para prover funcionalidade reduzida \cite{eswmobile}. Embora o hardware de dispositivos móveis tenha avançado bastante nos últimos anos, dispositivos móveis ainda apresentam capacidade reduzida de processamento devido a limitações como o tamanho reduzido e capacidade limitada de refrigeração. Devido a essas e outras limitações e a grande variedade de dispositivos Android no mercado, com poder computacional bem variado, aplicativos devem ser projetados para funcionar em hardware limitado. Em suma, deve-se pensar sempre em requisitos de performance e baixo consumo de recursos: uso de rede (3g/4g/wifi/bluetooth...), energia, ciclos de processamento, memória, entre outros. \citeonline{eswissues} afirma que o sucesso de qualquer aplicação, \textit{mobile} ou não, depende de uma grande lista de requisitos não funcionais.  

Segundo \citeonline{eswmobile}, deve-se analisar bem requisitos de contexto de execução de aplicativos dispositivos móveis. Aplicações móveis apresentam contextos de execução que não eram obtidos em tecnologias anteriores, com dados adicionais como localização, proximidade a outros dispositivos, entre outros, que podem alterar a forma com que os aplicativos são utilizados. Aplicativos móveis tem que ser pensados para se adaptar com essas mudanças de contexto.

O sistema Android permite checar a disponibilidade de recursos de hardware em tempo de instalação ou execução para que o desenvolvedor possa ajustar as funcionalidades apresentadas ao usuário e prevenir que o usuário encontre problemas na utilização de determinadas funcionalidades. Por exemplo, um jogo simples como um \textit{tic tac toe} que utilize \textit{bluetooth} para \textit{multiplayer} pode desativar essa funcionalidade para dispositivos antigos que o não tenham disponível e trabalhar apenas com jogo \textit{single player} contra algum tipo de jogador virtual. Da mesma forma, a ausência de algum recurso pode impedir que algum aplicativo seja instalado no dispositivo. O whatsapp, por exemplo, não pode ser instalado em dispositivos que não possuem comunicação com rede móvel via cartão SIM, como tablets que possuem apenas comunicação WIFI. Dessa forma é possível prevenir a apresentação para o usuário de funcionalidades que ele na verdade não pode executar.

Requisitos de software podem ser representados de diversas formas, sendo possível a utilização de vários modelos distintos. Na metodologia ágil scrum, por exemplo, os requisitos normalmente são registrados na forma de \textit{User Stories}, onde são geralmente descritos na visão do usuário do sistema. Em outros contextos, podem ser descritos em casos de uso, com descrições textuais e diagramas, ou outras várias formas de representação.

Requisitos de software geralmente tem como fonte o próprio cliente que contrata o serviço do desenvolvimento de software, e são extraídos da descrição de como esse cliente vê o uso do sistema. Todo esse processo é muitas vezes chamado de ``engenharia de requisitos''. 

Essa diferenciação que pode ser observada em requisitos para plataformas móveis em relação a software convencional também é refletida nas outras fases de desenvolvimento. Um produto idealizado de forma diferente acarreta em um produto trabalhado totalmente de forma diferente. As possíveis interações entre usuário e sistema, ou entre usuários, dão novos contextos de utilização para o produto de software. A forma como os usuários utilizam o próprio sistema alvo do software sendo desenvolvido muda a forma com que atividades de criação e inovação, planejamento, desenvolvimento, implantação e até mesmo distribuição e marketing são conduzidas.

\subsection{Desenho}

Desenho de software é o processo de definição da arquitetura, componentes, interfaces, e outras características de um sistema ou um componente \cite{swebok}. É durante o desenho de software que os requisitos são traduzidos na estrutura que dará base ao software sendo desenvolvido. As necessidades então são traduzidas em modelos, que descrevem os componentes e as interfaces entre os componentes. A partir desses modelos é possível avaliar a validade da solução desenhada e as restrições associadas a mesma, sendo possível avaliar diferentes soluções antes da implementação do software. A partir do design da arquitetura, é possível prever se alguns requisitos elicitados podem ou não ser atingidos com determinada solução, e mudá-la conforme necessário com pouco ou mesmo nenhum custo adicional.

A área de desenho de software pode variar conforme a tecnologia sendo utilizada. A arquitetura do sistema pode variar conforme o sistema operacional alvo, ou mesmo conforme a linguagem de programação que se está utilizando no desenvolvimento. Existem vários princípios de design de software amplamente conhecidos que se aplicam a uma imensidade de situações, sempre com o intuito de encontrar a melhor solução para cada situação e deixar o software modularizado e manutenível.

Aplicativos para o sistema Android são construídos em módulos, utilizando os componentes da API, embora possam ser criadas classes em Java puro sem a utilização de nenhum recurso da API do sistema, e utilizá-las nos componentes assim como em uma aplicação Java padrão desenvolvida para \textit{desktop}. Várias classes de modelo em uma arquitetura MVC (\textit{Model-View-Controller}), por exemplo, possivelmente serão criadas em Java puro. Por outro lado, o Android não impõe nenhuma arquitetura específica no desenvolvimento de aplicações, deixando livre para o desenvolvedor fazer suas escolhas.

\citeonline{androidmvc} apresentam alguns tipos de arquitetura derivados do bem difundido MVC, e demonstram uma possibilidade de adaptação do MVC ao Android. Embora a arquitetura MVC possa ser utilizada no Android, ela não é facilmente identificada, e não é intuitiva de ser implementada. \textit{Activities} são os componentes mais difíceis de serem encaixados na arquitetura MVC padrão, embora sejam bem adaptadas às necessidades do desenvolvedor. Por padrão elas tem responsabilidades correspondentes ao \textit{Controller} e ao \textit{View}, e são interpretadas de forma diferente por vários desenvolvedores para o MVC.

O Android provê um \textit{framework} para desenvolver aplicativos baseado nos componentes descritos no início deste capítulo. Os aplicativos são construídos com qualquer combinação desses componentes, que podem ser utilizados individualmente, sem a presença dos demais. Cada um dos componentes pode ser uma entrada para o aplicativo sendo desenvolvido. De forma geral, para desenhar uma arquitetura para sistema Android deve-se levar em conta todos esses componentes e conhecer bem sua aplicabilidade e a comunicação entre os mesmos. Mais detalhes sobre componentes Android podem ser encontrados no apêndice deste trabalho.

Neste trabalho a arquitetura do sistema e de aplicativos será avaliada e comparada através do resultado de métricas estáticas de código fonte, essencialmente métricas OO, que refletem várias das decisões arquiteturais. Como será discutido no Capítulo~\ref{cap:metodologia}, a principal questão de pesquisa é desenvolvida em cima da relação entre a API do sistema android e os aplicativos desenvolvidos para a mesma, relação que pode ser refletida em métricas estáticas de código como será discutido no Capítulo~\ref{cap:metricas}.

\subsection{Construção}

Essa área de conhecimento é responsável pela codificação do software, por transformar a arquitetura desenhada e seus modelos em código fonte. A construção de software está extremamente ligada ao desenho e às atividades de teste, partindo do primeiro e gerando insumo para o segundo \cite{swebok}.

Várias medidas podem ser coletadas do próprio código pra auxiliar a avaliação da qualidade do produto sendo construído e gerar insumo para o próprio desenvolvedor reavaliar sua implementação antes da fase de testes. 

Este trabalho visa auxiliar a fase de construção de aplicativos Android através da análise de métricas estáticas de código que refletem o design nesse contexto de desenvolvimento Android. O objetivo dessa análise é auxiliar desenvolvedores de aplicativos nos primeiros estágios de desenvolvimento e continuamente durante a construção do produto, provendo uma avaliação do estado atual e uma base de comparação para o projeto durante todo o ciclo de vida, trabalhada a partir do código do sistema e de aplicativos nativos, como email, calendário, contatos, câmera, calculadora e o \textit{web browser}.

Essa base de comparação deve ser idealizada como uma referência válida para este contexto de desenvolvimento de aplicativos, então as conclusões tiradas para o código do sistema devem se mostrar válidas também para aplicativos desenvolvidos para o mesmo. O acoplamento de aplicativos com a própria API do sistema indica que isso é uma possibilidade bastante plausível, o que será discutido no Capítulo~\ref{cap:resultados}. 

\subsection{Manutenção}

A manutenção de software trata dos esforços de desenvolvimento com o software já em produção, isto é, em funcionamento no seu devido ambiente. Problemas que passaram despercebidos durante as fases de construção e testes são encontrados e corrigidos durante a manutenção do sistema. Da mesma forma, o usuário pode requisitar novas funcionalidades que ele não havia pensado antes do uso do sistema, e o desenvolvimento dessas novas funcionalidades é também tratado como manutenção uma vez que o software já se encontra em produção, um processo conhecido como evolução de software. Revisões de código e esforços com manutenibilidade também podem ser consideradas atividades de manutenção, embora possam acontecer antes do sistema entrar em produção.

A manutenção de software geralmente acontece por período mais longo que as demais fases do desenvolvimento do software citadas nos tópicos anteriores, ocupando a maior parte do ciclo de vida do produto.

A separação de componentes independentes apresentados neste capítulo é de grande valia para a manutenibilidade do sistema. Essa separação permite que componentes tenham sua funcionalidade específica melhor compreendida e possam ser substituídos sem grandes impactos nos demais. 

Atividades de design de software devem ser reforçadas para garantir uma fase de manutenção com a menor quantidade de problemas possível. Uma arquitetura modularizada é mais fácil de ser entendida e modificada e consequentemente mantida. Também é importante que se utilize de padrões de codificação, identação e documentação para que a manutenção seja facilitada inclusive para desenvolvedores que não tem familiaridade com o código desenvolvido. Empresas tendem a colocar equivocadamente engenheiros junior para dar manutenção a sistemas e engenheiros mais experientes para desenvolver novos projetos, e adotar práticas de desenvolvimento que agem em favor à manutenibilidade ajudam a amenizar os problemas causados por esse tipo de alocação de recursos humanos.

Os resultados de métricas que refletem decisões arquiteturais geralmente tem relação com a manutenibilidade do software. Alto acoplamento entre objetos, por exemplo, indica que uma mudança em um pequeno trecho de código pode trazer resultados catastróficos no restante do software.

Sendo um sistema de código aberto, o Android permite que o desenvolvedor possa o analisar e consequentemente o entender a ponto de corrigir \textit{bugs} do sistema, criar funcionalidades novas, e também portá-lo para novos \textit{hardwares} \cite{googleandroid}. Isso é um ponto importante para atividades de manutenção do sistema Android como um todo, e permitiu que a plataforma crescesse de forma extremamente acelerada em um pequeno espaço de tempo. Um dos motivos para a própria plataforma Linux estar em um estado tão estável nos dias atuais foi os anos que a mesma teve de contribuição da comunidade sendo um software livre e portanto tendo seu código aberto a qualquer desenvolvedor, assim como o sistema Android.

Inserido no contexto de manutenção e evolução do sistema Android, este trabalho avalia resultado de métricas em um sistema operacional consolidado e amplamente utilizado para auxiliar qualquer etapa que envolva manipulação de código fonte no ciclo de desenvolvimento de outros projetos relacionados para esta plataforma. Desenvolvedores poderão utilizar resultados deste estudo para tomar decisões relacionadas a remodelagem e refatoração de seus projetos que já estejam em produção, a fim de melhorar sua qualidade e facilitar sua evolução.

A atividade de testes, descrita no apêndice deste documento, corresponde a outra atividade essencial no contexto de manutenção de um produto de software. É importante ressaltar que, assim como podem refletir a qualidade de um produto de software, métricas de código fonte podem ajudar a prever esforços de teste já na fase de construção. A complexidade ciclomática em valores muito altos, por exemplo, pode indicar um software praticamente intestável, e portanto muito difícil de se manter.

Em linhas gerais, impactos positivos podem ser vistos em todas as etapas do ciclo de vida do software quando se monitora o desenvolvimento do produto desde sua concepção.