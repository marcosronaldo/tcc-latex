\chapter{Considerações finais}
\label{consideracoes-finais}

Grande parte da literatura ainda referencia diretamente o guia de desenvolvimento da plataforma Android, e não há contribuições significativas que já não estejam contempladas no próprio guia. Por esse motivo, várias fontes de leitura apenas apresentam informações redundantes e portanto não foram referenciadas neste documento. As principais fontes aqui referenciadas contemplam trabalhos de pesquisa em segurança, estudo e análise de arquitetura, ou outras temáticas relacionadas não apenas ao Android, mas à dispositivos móveis em geral e à engenharia de software.

Como plataforma de desenvolvimento, o Android apresenta um ambiente bastante facilitado, com uma curva de aprendizado relativamente rápida para desenvolvedores familiarizados com linguagem de programação Java. Sua extensa documentação online, com o guia de desenvolvimento bastante detalhado e várias comunidades de desenvolvedores, como o \textit{stackOverflow}, que vem crescendo em uma rápida proporção juntamente com o avanço da própria plataforma, apresentam aos novos desenvolvedores uma forma rápida de aprender com os desenvolvedores mais experientes e também com os próprios criadores da plataforma, que também participam de fóruns de desenvolvedores e ajudam a esclarecer diversos pontos sobre a utilização da API. Esse imenso suporte que é encontrado online é decisivo para a escolha dessa plataforma de desenvolvimento, e ao longo deste trabalho exerceu um papel importantíssimo para o avanço da equipe.

Embora apresente ambiente favorável ao aprendizado, alguns problemas foram encontrados pela equipe durante este período de desenvolvimento. A maioria deles foi relacionado a testes dentro da plataforma, que se apresentaram difíceis de serem implementados para a utilização de determinados recursos da API. Houve problemas com a simulação da comunicação entre os componentes, e mesmo uma grande quantidade de esforço direcionada a esse fim ao longo de várias \textit{sprints} não resultou em avanço nessa direção. O desenvolvimento ao longo do semestre foi bastante limitado no que diz respeito aos testes dos próprios componentes internos do aplicativo, o que é imprescindível para o desenvolvimento de um produto de qualidade. Dessa forma, uma das tarefas e provável desafio para a próxima etapa deste trabalho é realizar os testes de todos os componentes internos do aplicativo e-lastic, bem como testes de aceitação para a interface gráfica que será entregue também nesse trabalho futuro. 

Embora apenas alguns testes internos tenham sido implementados com sucesso, a equipe obteve bons resultados na implementação de testes de interação com usuário, utilizando das ferramentas \textit{calabash}~\footnote{Para mais informações visite \url{http://calaba.sh/}} e  \textit{cucumber}.

Devido a equipe mais inexperiente no desenvolvimento Android, grande parte do esforço neste período de trabalho foi direcionado ao treinamento da equipe e participação dentro do ciclo de desenvolvimento. Isso limitou um pouco o tempo dedicado a parte teórica  de estudo e levantamento bibliográfico deste trabalho, embora tenha alcançado resultados importantes no desenvolvimento. É importante ressaltar que esse não era o objetivo inicial de trabalho, que deveria ter sido mais focado na revisão de código e no papel de consultor da plataforma em vez de implementador de fato. Com isso, houve algumas etapas com maior intervenção dentro do ciclo de desenvolvimento e outras onde a equipe levou o desenvolvimento por sua conta enquanto meu papel de revisor e consultor era mantido.

Esta etapa de desenvolvimento, focada na arquitetura do aplicativo, alcançou um resultado satisfatório nos quesitos de manutenibilidade e modularização, gerando uma sólida estrutura arquitetural que será o ponto de partida para a próxima etapa deste trabalho, que terá como foco a finalização do desenvolvimento e entrega do produto de software.
