\chapter{Introdução}
\section{Contexto}

%A computação eletrônica tem avançado muito desde o século XX, e esse avanço continua bastante visível no século XXI \cite{histcomputacao2007}. Aparelhos e maquinas que antes custavam fortunas e ocupavam salas inteiras puderam avançar ao ponto de caber em uma única caixa. Engrenagens e inúmeras válvulas reduzidas a um micro chip. Várias ferramentas utilizadas no dia a dia de muitos trabalhadores reunidas em um único computador, que adquiriu muitas outras funções além de fazer cálculos.

Nos últimos anos, muitas das funções de um computador vem sendo gradativamente transferidas para dispositivos que cabem na palma da mão \cite{mobilemassmedia}. Atualmente podemos encontrar dispositivos móveis com poderosos processadores, alta capacidade de armazenamento de dados e conectividade sem fio avançada. A cada dia temos uma maior quantidade de informação concentrada de forma prática e segura, resultando em tarefas cotidianas automatizadas e controladas.

Para trabalhar nesses dispositivos, precisamos saber como eles funcionam e como podemos utilizá-los, para criar produtos de alta qualidade sem impactar no desempenho do processo de desenvolvimento. Monitorar o desenvolvimento desses produtos ao longo do seu desenvolvimento pode impactar bastante na qualidade de um produto final, e métricas de código fonte podem ser utilizadas em todas as etapas da criação de um software com esse objetivo. 

Vários estudos já foram conduzidos sobre a qualidade de um produto de software. \citeonline{ooasqualityindicators} apresenta um estudo prático demonstrando que métricas de código fonte podem ser bastante úteis como indicadores de qualidade. \citeonline{qualitypredictionsvm} apresentam um método para prever qualidade de software em estágios iniciais de desenvolvimento baseado em métricas de complexidade de código. Vários outros estudos semelhantes já foram conduzidos nessa mesma direção. \citeonline{meirelles2013} afirma que a avaliação de qualidade de código fonte no início do desenvolvimento pode ser de grande valia para auxiliar equipes inexperientes durante as etapas seguintes do desenvolvimento de software.

Poucos trabalhos podem ser encontrados na literatura com aplicação de métricas de código fonte na API (\textit{Application Programming Interface}) Android ou em seus aplicativos, e dado o contexto atual de disseminação de \textit{smartphones} e \textit{tablets}, o foco desse trabalho está em auxiliar na qualidade de produtos de software desenvolvidos para uma das maiores plataformas móveis da atualidade, o sistema operacional Android. 

\section{Objetivos}

O objetivo principal deste trabalho é o monitoramento de métricas de código fonte na API do sistema operacional Android, essencialmente métricas orientadas a objetos, e fazer um estudo da evolução de seus valores nas diferentes versões da API, estudar as semelhanças com aplicativos do sistema e então verificar a possibilidade de utilizar os dados obtidos para auxiliar no desenvolvimento de aplicativos. Será apresentada uma proposta de cálculo de similaridade entre aplicativos e a API Android, utilizando como exemplo de uso desse cálculo um aplicativo parcial desenvolvido sem monitoramento de métricas.

Objetivos específicos:
\begin{itemize}
\item Estudo da evolução de métricas de código fonte em diversas versões da API;
\item Estudo da correlação da API Android e de seus aplicativos;
\item Definição e validação de intervalos de referência para valores de métricas;
\item Proposta de utilização dos intervalos definidos para auxílio no desenvolvimento de aplicativos;
\end{itemize}

\section{Estrutura do trabalho}

Este documento está dividido em 4 capítulos. No Capítulo~\ref{cap:metodologia} são melhor explicados os objetivos do trabalho e os métodos para o alcance dos objetivos propostos. Em seguida, o Capítulo~\ref{cap:analise_exploratoria} apresenta discussões sobre análise dos dados coletados, validação dos mesmos, e uma proposta de aplicação dos valores definidos. O Capítulo~\ref{cap:elastic} apresenta um aplicativo Android parcial com base em um estudo de caso específico, desenvolvido neste trabalho com o objetivo de alcançar uma boa arquitetura a partir de padrões de projeto e sem o auxílio de métricas, para base de comparação. Por fim, são apresentadas as considerações finais sobre a análise dos dados coletados e verificação de similaridade, bem como sugestões para continuidade deste estudo.
