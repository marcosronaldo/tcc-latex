\chapter{Resultados}
\label{cap:resultados}

%Neste capítulo serão discutidos os resultados da análise dos dados coletados, bem como resultados referentes ao modelo de regressão criado para utilizar os dados como um auxílio para desenvolvimento de projetos futuros.
%TODO criar explicação geral para o capítulo

\section{Análise preliminar}

Com os dados coletados e devidamente preparados, várias conclusões podem ser tiradas dos valores das métricas e sua evolução ao longo do tempo. Esta seção é focada na análise subjetiva dos dados, tentando explicar seu comportamento com relação às características do sistema, compará-los a outros estudos, e até mesmo comparar com dados de métricas em aplicativos, utilizando os próprios aplicativos do sistema como base de comparação.

Embora seja relevante comentar as diferenciações entre as linguagens e seus paradigmas para algumas métricas,  não há necessidade de separação entre os valores para linguagem C, procedural, e linguagens Java/C++, orientadas a objetos, uma vez que, dadas as proporções das mesmas apresentadas nos dados, não há relevância estatística para tal. Entretanto em algumas métricas algumas observações teóricas possam ser ressaltadas, embora, mais uma vez, não haja implicação substancial no resultados gerais apresentados.

\subsection{Average Method Lines Of Code}

A primeira observação que deve ser feita quando analisando LOC nesse contexto é a diferenciação das linguagens. Embora um módulo em C seja mapeado para uma classe, arquivos fonte em C tendem a ser maiores que uma classe em Java, por exemplo, devido aos diferentes paradigmas que essas linguagem utilizam. Arquivos em C++ e Java também podem ter valores bem distintos para a mesma funcionalidade devido ao número de bibliotecas padrões que a linguagem apresenta e a natureza da própria sintaxe da linguagem. Dessa forma, comparações dessa métrica devem ser feitas somente dentro da mesma linguagem. Neste trabalho não serão feitas comparações diretas dos valores desta métrica, então ela será utilizada principalmente para relativização da comparação de outras métricas, quando aplicável, uma vez que os valores de algumas delas podem ser relacionados a esta. 

A métrica LOC por si só não será discutida aqui, pois seu valor é mais relativo e deve ser comparado com outras métricas para ter significado mais completo. Uma classe com valor alto de LOC pode ter um baixo valor de AMLOC e valor maior para NOM, ainda mantendo um valor aceitável de LCOM. Em suma, a análise de outras métricas abrange as explicações relacionadas a métrica LOC e também a NOM, então essas métricas de tamanho não serão explanadas em separado, mas juntamente com a explicação de outras métricas.

Valores baixos de AMLOC são sempre preferíveis pois métodos mais enxutos tem menor responsabilidade, portanto estão mais sujeitos a reuso, e também são mais fáceis de se ler e se modificar. Entretanto essa métrica é preciso ser analisada em conjunto com outras métricas, como LCOM e RFC. Uma classe com muitos métodos privados pequenos tende a ter um valor maior de RFC, o que não implica que esteja mal projetada, desde que os métodos ali presentes estejam bem posicionados segundo o padrão de projeto OO especialista da informação, mantendo por consequencia um baixo valor de LCOM. Como referência geral de resultados para AMLOC, quanto menor o valor, melhor o resultado.

\begin{table}[!htb]
\scalefont{.7}
\documentclass[a4paper]{article}
\usepackage[T1]{fontenc}
\usepackage[latin1]{inputenc}
\begin{document}
\begin{tabular}{|l|l|l|l|l|l|l|l|l|l|l|l|}
\hline
version&min&1\%&5\%&10\%&25\%&50\%&75\%&90\%&95\%&99\%&max\\
\hline
android-1.6\_r1.2&0&0&0&0&2.33333333333333&5.57142857142857&11.5&21.5&30&65.8742857142858&312\\
\hline
android-1.6\_r1.5&0&0&0&0&2.33333333333333&5.57142857142857&11.5&21.5&30&65.8742857142858&312\\
\hline
android-2.0\_r1&0&0&0&0&2&5.55555555555556&11.5&21.8461538461538&30.1916666666666&67.8124999999999&390.5\\
\hline
android-2.1\_r2.1p2&0&0&0&0&2&5.625&11.5&21.8571428571429&30.3366666666666&68.4017499999999&395\\
\hline
android-2.2\_r1&0&0&0&0&1.75892857142857&5.8&12.8&26.5&44.2135714285714&156.66&1034\\
\hline
android-2.2.3\_r2&0&0&0&0&1.76587301587301&5.81666666666666&12.8214285714286&26.5&44.1664285714285&156.62&1034\\
\hline
android-2.3\_r1&0&0&0&0&1&5.8&13.6&30.1787878787879&55.3636363636363&164.773333333334&1034\\
\hline
android-2.3.7\_r1&0&0&0&0&1&5.83333333333333&13.7064950980392&30&54.0590277777778&163.397272727272&1034\\
\hline
android-4.0.1\_r1&0&0&0&0&1&5.85714285714286&14&31&54.3731481481482&162.418181818181&1034\\
\hline
android-4.0.4\_r2.1&0&0&0&0&1&5.85714285714286&14&31&53.9791666666667&162&1034\\
\hline
\end{tabular}
\end{document}

\scalefont{.7}
\caption{Percentis para a métrica \textit{Average Method Lines of Code} no Android}
\label{tab:amloc_android}
\end{table}

A Tabela~\ref{tab:amloc_android} apresenta os valores para a métrica AMLOC nas versões do Android analisadas. É facilmente perceptível que a média de linhas de código por método não teve variação relevante. Em todas as versões analisadas, os valores muito frequentes, isto é, percentil 75, são métodos com até 14 linhas de código, enquanto de 14 a 30 aparecem como frequentes, e 31 a 55 pouco frequentes. Esses são valores que estão de acordo com os apresentados em \citeonline{meirelles2013}, porém levemente menores para os percentis 75 e 90, com aproximadamente 3 linhas de código a menos por método. É possível perceber que os valores se mostraram bem semelhantes para o projeto Android, mesmo considerando o fato que este trabalho estuda apenas a API de desenvolvimento de aplicativos, essencialmente em java e dentro do diretório ``\textit{frameworks}'' do AOSP, e \citeonline{meirelles2013} analiza todo o código fonte do sistema, que apresenta em sua totalidade uma maior proporção da linguagem C em relação as demais. Esses valores são subsídios para reafirmar que arquivos em C em geral, tem uma maior utilização de linhas de código do que arquivos em Java. \citeonline{oliveira2013} comenta que as diferenças entre as linguagems C/C++/Java para esta métrica não é significativa, uma vez que a sintaxe entre as 3 é bastante semelhante. Dada essa afirmação, podemos comparar os intervalos definidos por ele, chegando a conclusão de que os valores das métricas estão, para todas as versões, abaixo dos valores e regular para os percentis 75 e 90, o que é um bom resultado.

\begin{table}[!htb]
\scalefont{.7}
\begin{tabular}{|l|l|l|l|l|l|l|l|l|l|l|l|l|}
\hline
app&classes&min&1\%&5\%&10\%&25\%&50\%&75\%&90\%&95\%&99\%&max\\
\hline
Launcher2&161&0&0&1&1&3&6.33&10.78&16.9&24.38&48.36&57.5\\
\hline
Settings&722&0&0&0&0&3.63&8&15&21.47&28.5&49.4&80.42\\
\hline
Camera2&462&0&0&0&0&1&4.5&9.85&16.09&21.5&38.28&66.67\\
\hline
Bluetooth&239&0&0&0&2.83&6.02&10.79&22.16&39.86&59.84&111.11&221\\
\hline
QuickSearchBox&196&0&0&1&1&3&4.39&6.17&10.53&13.15&22.28&32\\
\hline
Calculator&10&1&1&1&1&6.67&8.92&13.33&23.26&27.38&30.68&31.5\\
\hline
Terminal&17&0&0.15&0.75&1.85&3.09&8.12&15.29&20.92&29.38&48.27&53\\
\hline
PackageInstaller&19&0&0.68&3.4&4&4.63&6.59&16.98&18.9&22.89&31.45&33.59\\
\hline
Dialer&215&0&0&0&1&3&7&11.13&16.89&19.98&32.02&61.33\\
\hline
Browser&259&0&0&1&1&3.5&6.89&11&19&25.95&46.02&55.33\\
\hline
InCallUI&117&0&0&0&0&1&4.23&12&18.75&23.32&40.77&58\\
\hline
LegacyCamera&214&0&0&0&0.2&4&8.64&15.78&25.47&32.18&69.42&112.67\\
\hline
Gallery2&895&0&0&0&0&3&6&11.5&17.38&21.67&44.36&107\\
\hline
BasicSmsReceiver&5&8.67&8.83&9.47&10.27&12.67&16.33&18.75&18.9&18.95&18.99&19\\
\hline
UnifiedEmail&872&0&0&0&1&3&5&9.71&17&23.67&37.95&139.63\\
\hline
Launcher3&354&0&0&0&0&2.73&5.13&10.59&17.17&24.79&54.71&163.5\\
\hline
Music&75&0&0&0&1&4.1&9.51&16.89&21.76&28.0&48.32&90\\
\hline
Camera&253&0&0&0&1&3&7.42&13&22.37&31.45&72.23&112.67\\
\hline
Email&400&0&0&0&1&3.58&8&15.35&24.49&31.61&63.28&128\\
\hline
Nfc&178&0&0&0&1&3&9.64&18.5&31.63&38&42.48&70.5\\
\hline
Gallery&89&0&0.87&1&1&4&7.63&12.67&19.0&28.6&53.12&55\\
\hline
ContactsCommon&292&0&0&0&1&3.23&7.1&13&19&23.88&34.5&53.33\\
\hline
Contacts&265&0&0&0&1&3&6.45&11.5&18.61&23.72&63.53&86\\
\hline
DeskClock&121&0&0&0&1&5&9.16&15.26&24.02&27.3&30.71&40.13\\
\hline
HTMLViewer&4&5&5.12&5.6&6.2&8&11&14.5&16.6&17.3&17.86&18\\
\hline
Calendar&216&0&0&0&1&5&11.67&19.58&30.95&39.3&90&115.5\\
\hline
Exchange&135&0&0&0&1&4&10.01&17.31&28.41&34.65&44.41&51.25\\
\hline
\end{tabular}

\scalefont{.7}
\caption{Percentis para a métrica \textit{Average Method Lines of Code} nos aplicativos nativos}
\label{tab:amloc_apps}
\end{table}

Os valores apresentados na análise são relativamente baixos quando comparados com outros softwares livres, como demonstrado por \citeonline{meirelles2013}. Da mesma forma, quando olhamos os valores aplicativos do sistema, demonstrados na Tabela~\ref{tab:amloc_apps}, podemos perceber uma grande semelhança nos resultados. Embora alguns poucos aplicativos tenham valores mais elevados para essa métrica, pode-se perceber que os intervalos se mantém válidos para a grande maioria dos aplicativos. Esses valores de aplicativos foram retirados dos aplicativos nativos da ultima versão do sistema analisada (Lollipop 5.1.0), e continuam se mantendo semelhantes ao sistema, como o próprio acoplamento à API sugere.

Em linhas gerais, os aplicativos do sistema também se mantém dentro dos intervalos bom e regular definidos em \citeonline{oliveira2013}. Os valores para o percentil 95 também se encontram abaixo do valor regular, na maioria dos casos.

Em suma, os valores para os aplicativos se assemelham muito com os valores para as versões da API Android analisadas, levando então a conclusão de que os mesmos intervalos são válidos para as métricas em ambos os casos, embora se possa esperar valores menores em aplicativos, porem com uma maior variância. Essa variância se dá pelo diferente propósito de cada aplicativo, que utiliza pedaços variados do sistema e tem sua codificação adaptada para seu propósito.

Intervalos encontrados:

\begin{itemize}
\item Valores abaixo de 14 se mostraram muito frequentes para os aplicativos e para a API;
\item Enquanto no sistema os valores para o percentil 90 se encontram abaixo de 31, nos aplicativos eles alcançam em poucos casos, ficando em sua maioria abaixo de 25;
\item Valores acima de 31 são pouco frequentes em ambos os casos;
\end{itemize}

\subsection{Average Cyclomatic Complexity per Method}

Complexidade ciclomática nada mais é do que o número de caminhos que um software pode seguir dada uma execução qualquer. Na prática, cada condicional dentro do sistema incrementa o valor desta métrica em 1, uma vez que divide a execução em um caminho de execução se a expressão condicional for válida, ou um segundo caminho caso não seja. Complexidade ciclomática é calculada a nível de método, e o valor de ACCM para uma classe corresponde a média dos valores de complexidade ciclomática de cada um dos seus métodos.

A interpretação do valor de complexidade ciclomática é relativamente simples: O valor 1 é o valor mínimo e ideal para se ter como resultado, pois significa que o software tem apenas uma forma de executar e será executado necessariamente daquela forma e naquela sequência. Como consequência disso, se tem um software que pode ser mais facilmente lido e modificado. A implicação dessa métrica é mais notada na atividade de testes do código fonte, pois além de dificultar a compreensão dos possíveis comportamentos de um pedaço de código, cada caminho adicional que pode ser seguido é um trecho diferenciado que deve ser testado. Isso quer dizer que o esforço de teste é diretamente proporcional ao resultado dessa métrica, pois para garantir o funcionamento correto do sistema, todos as possibilidades devem ser devidamente testadas. Em termos práticos, atingir uma cobertura de código de 100\% é uma tarefa árdua quando há um valor muito grande de complexidade ciclomática.

\begin{table}[!htb]
\scalefont{.7}
\begin{tabular}{|l|l|l|l|l|l|l|l|l|l|l|l|l|}
\hline
versão&classes&min&1\%&5\%&10\%&25\%&50\%&75\%&90\%&95\%&99\%&max\\
\hline
android-1.6\_r1.2&5745&0&0&0&0&1&1.11&2&3.45&4.69&9.5&55\\
\hline
android-1.6\_r1.5&5745&0&0&0&0&1&1.11&2&3.45&4.69&9.5&55\\
\hline
android-2.0\_r1&6331&0&0&0&0&1&1.11&2&3.5&4.75&9.74&59\\
\hline
android-2.1\_r2.1p2&6360&0&0&0&0&1&1.12&2&3.5&4.8&9.88&60\\
\hline
android-2.2\_r1&7352&0&0&0&0&1&1.07&2&3.74&5.28&12&99\\
\hline
android-2.2.3\_r2&7358&0&0&0&0&1&1.07&2.02&3.75&5.26&12&99\\
\hline
android-2.3\_r1&8093&0&0&0&0&1&1&2.07&4&5.82&12.83&99\\
\hline
android-2.3.7\_r1&8240&0&0&0&0&1&1&2.08&4&5.8&12.76&99\\
\hline
android-4.0.1\_r1&11709&0&0&0&0&1&1&2.13&4&6&17&94.33\\
\hline
android-4.0.4\_r2.1&11851&0&0&0&0&1&1&2.11&4&6&17&94.33\\
\hline
android-4.1.1\_r1&14115&0&0&0&0&1&1&2&3.86&5.78&16&99.4\\
\hline
android-4.3.1\_r1&15472&0&0&0&0&1&1&2&3.62&5.23&12&120.4\\
\hline
android-5.1.0\_r1&20129&0&0&0&0&1&1&2&3.5&5&11&158.6\\
\hline
\end{tabular}

\scalefont{.7}
\caption{Percentis para a métrica \textit{Average Cyclomatic Complexity per Method} no Android}
\label{tab:accm_android}
\end{table}

Inserida então no contexto de manutenção e testes, essa métrica é uma excelente candidata para ser constantemente monitorada ao longo da evolução do código fonte. Embora não tenha muita relação com outras métricas OO, ACCM tem uma relação óbvia do número de linhas de código de um método, pois um método com poucas linhas de código não tem possibilidade de ter um valor muito alto de complexidade ciclomática. De forma geral, métodos grandes ``abrem espaço'' para problemas de complexidade excessiva, como já comentado na seção anterior. Essa relação pode ser claramente vista na Tabela~\ref{tab:accm_android}. Nos valores do percentil 75, que correspondem a valores muito frequentes, são os únicos valores para o sistema onde essa métrica supera o número 2, e não por acaso são os valores com maior AMLOC nesse percentil como pode ser visto na Tabela~\ref{tab:amloc_android}. Essa relação direta também pode ser vista nos percentis 90 e 95, que representam valores frequentes e pouco requentes, respectivamente.

\begin{table}[!htb]
\scalefont{.7}
\documentclass[a4paper]{article}
\usepackage[T1]{fontenc}
\usepackage[latin1]{inputenc}
\begin{document}
\begin{tabular}{|l|l|l|l|l|l|l|l|l|l|l|l|}
\hline
app&min&1\%&5\%&10\%&25\%&50\%&75\%&90\%&95\%&99\%&max\\
\hline
Launcher2&0&0&1&1&1&1.4&2.09318181818182&3&3.61545454545454&8.67642857142855&11.8666666666667\\
\hline
Settings&0&0&0&0&1&1.57142857142857&2.5&3.66666666666667&4.42857142857143&8&17\\
\hline
Camera2&0&0&0&0&1&1&1.83333333333333&2.66666666666667&3.5&6&8.3\\
\hline
Bluetooth&0&0&0&1&1.33333333333333&2.19337979094077&4&7.95&10.3625&24.2866666666666&49\\
\hline
VoiceDialer&1&1&1&1&1.75&3.5&6.16666666666667&8.52&8.93999999999999&11.388&12\\
\hline
QuickSearchBox&0&0&1&1&1&1.03333333333333&1.66666666666667&2.47142857142857&3&4.67666666666667&5\\
\hline
Calculator&1&1&1&1&1.33333333333333&1.58333333333333&2.66666666666667&3.53333333333333&5.26666666666667&6.65333333333333&7\\
\hline
Mms&0&0&0&0&1&1.28571428571429&2.43650793650793&3.55555555555556&5.4125&10.2285714285714&16\\
\hline
ManagedProvisioning&1&1&1&1&1&2.16764705882353&2.98076923076923&3.24&3.575&5.26000000000001&6\\
\hline
SoundRecorder&1&1.01714285714286&1.08571428571429&1.17142857142857&1.42857142857143&1.6&1.9047619047619&2.94372294372294&3.29004329004329&3.56709956709957&3.63636363636364\\
\hline
Terminal&0&0.15&0.75&1&1&1.5&1.73809523809524&4.8&8.25&10.45&11\\
\hline
PackageInstaller&0&0.17&0.85&1&1&1.2984126984127&2.86458333333333&3.7&4.07058823529412&4.39058823529412&4.47058823529412\\
\hline
SpareParts&1&1.01357142857143&1.06785714285714&1.13571428571429&1.33928571428571&1.67857142857143&2.01785714285715&2.22142857142857&2.28928571428572&2.34357142857143&2.35714285714286\\
\hline
Tag&1&1&1&1&1.1&1.6&1.95833333333334&2.71666666666667&2.975&3.395&3.5\\
\hline
CertInstaller&1&1&1&1&1&1.57142857142857&2.40384615384616&2.7&3.35&3.87&4\\
\hline
KeyChain&0&0.17&0.85&1&1.08333333333333&1.66666666666667&2&2.73&3.51875&3.60375&3.625\\
\hline
Dialer&0&0&0&1&1&1.16666666666667&2&2.77142857142857&3&3.9675&57.8333333333333\\
\hline
Browser&0&0&1&1&1&1.5&2.16666666666667&3.26896551724138&4.03749999999999&7.02500000000001&8.8\\
\hline
PhoneCommon&0&0.17&0.85&1&1&1.16666666666666&1.75&3.51666666666667&5.6125&6.1225&6.25\\
\hline
InCallUI&0&0&0&0&1&1.05555555555555&2.00595238095238&2.76318681318681&4&6.69999999999999&8.33333333333333\\
\hline
CellBroadcastReceiver&1&1&1&1&1.2390350877193&1.76623376623377&3.33928571428572&6.13&9.72&13.636&15\\
\hline
OneTimeInitializer&1&1.02&1.1&1.2&1.5&2&2.5&2.8&2.9&2.98&3\\
\hline
LegacyCamera&0&0&0&0.200000000000003&1&1.6&2.39655172413793&3.36666666666667&4.10599078341013&9.58666666666667&10\\
\hline
Gallery2&0&0&0&0&1&1.375&2.16176470588236&3&3.705&6.01399999999999&11\\
\hline
BasicSmsReceiver&1.33333333333333&1.34333333333333&1.38333333333333&1.43333333333333&1.58333333333334&1.70833333333334&1.99431818181818&2.43409090909091&2.58068181818182&2.69795454545455&2.72727272727273\\
\hline
MusicFX&1&1&1&1&1&2&3.13166666666667&4.96521739130435&7.09782608695653&9.4195652173913&10\\
\hline
TvSettings&0&0&1&1&1&1.40689655172414&2.48026315789474&3.77417582417583&4.7125&6.39749999999998&12\\
\hline
Stk&1&1&1&1.2&2&2.5&3.5&4.724&7.09999999999999&13.82&15.5\\
\hline
UnifiedEmail&0&0&0&1&1&1.16666666666667&1.93541666666667&2.83333333333333&3.66666666666667&6.69166666666666&53\\
\hline
Launcher3&0&0&0&0&1&1.23076923076923&2&3&4&9.87212121212122&30\\
\hline
Music&0&0&0&1&1&1.66666666666667&2.5&3.37436440677966&4.31484517304189&8.96624999999995&18\\
\hline
Camera&0&0&0&1&1&1.49418604651163&2.26559829059829&3.16944801026957&3.96642857142857&9.83000000000001&17\\
\hline
Email&0&0&0&1&1&1.33333333333333&2&3.02222222222222&4.17721139430285&7.50999999999999&19.4\\
\hline
Nfc&0&0&0&1&1&2&3.35294117647059&5.16140350877193&7.86666666666666&9.62&15.5\\
\hline
Gallery&0&0.87&1&1&1&1.6125&2.5&3.35333333333333&3.9125&7.25999999999999&9\\
\hline
ContactsCommon&0&0&0&1&1&1.28571428571429&2&3.44444444444444&4.54166666666667&7&7.5\\
\hline
Contacts&0&0&0&1&1&1.23214285714285&2&3&3.64017857142857&9.74000000000001&21\\
\hline
DeskClock&0&0&0&1&1&1.6875&2.28928571428572&3.25833333333333&3.81&4.48642857142857&5.33333333333333\\
\hline
FMRadio&0&0&0&0&0&0.5&1.69602272727273&2.535&3.27298850574712&9.76500000000001&13.5\\
\hline
\end{tabular}
\newline
\begin{tabular}{|l|l|l|l|l|l|l|l|l|l|l|l|}
\hline
HTMLViewer&1.5&1.51&1.55&1.6&1.75&2&2&2&2&2&2\\
\hline
Calendar&0&0&0&1&1&2&3&4.68&6.325&14.93&19\\
\hline
Exchange&0&0&0&1&1&1.65526315789473&3.2&4.49&5.4112443778111&6.82623913043478&7.66666666666667\\
\hline
\end{tabular}
\end{document}

\scalefont{.7}
\caption{Percentis para a métrica \textit{Average Cyclomatic Complexity per Method} nos aplicativos nativos}
\label{tab:accm_apps}
\end{table}

%TODO criar e linkar gráfico de ACCM em função de AMLOC

Essa mesma relação pode ser vista para os aplicativos. Embora não seja totalmente determinístico, no geral aplicativos com maiores valores de AMLOC tendem a ter um maior valor de ACCM, assim como visto na API do sistema. Com exceção do \textit{SMSReceiver}, os 5 aplicativos com maior valor de AMLOC são os que contém maior complexidade do conjunto. Esses resultados para a métrica ACCM, que é uma métrica bastante difundida e tem sua aplicabilidade bem clara, dão subsídio para reafirmar que os valores da métrica AMLOC devem ser os menores possíveis para uma boa arquitetura orientada a objetos.

Entender essa relação é importante também para pensar em possibilidade de refatoração de um código fonte alvo. Estão claras as consequências de se ter uma alta complexidade ciclomática, mas entender sua relação com AMLOC nos leva rapidamente a ter a idéia verificar os métodos da classe e avaliar se seu comportamento está tão enxuto como deveria ser, se algum método não está fazendo mais do que propõe. As vezes dividir o comportamento em tarefas menores possa ser uma solução viável. Dividir o comportamento de um método em 2 provavelmente vai acarretar no aumento da métrica RFC, mas mais uma vez ressalto que o valor de RFC não deve ser analisado por si só. É importante ficar de olho também em métricas como a LCOM quando fazendo esse tipo de refatoração, pois as vezes uma tarefa menor que foi extraída de um método não está coesa na classe onde está e muitas vezes até já esteja implementada em uma outra classe que tem responsabilidade mais congruente com essa tarefa. Remover esse tipo de código duplicado ajuda a reduzir a falta de coesão dentro de uma classe, e todas essas melhorias derivaram do simples fato de perceber uma alta complexidade ciclomática em uma classe.

Na verdade, embora algumas métricas tenham um indicativo de um problema pontual, nenhuma das métricas discutidas neste trabalho deve ser analisada isoladamente.

Reforçando a importância dessa métrica em uma análise estática de código, \citeonline{oliveira2013} define a mesma com um peso adicional em relação a outras métricas em seu estudo. Valores de referencia definidos para esse estudo foram 1 a 3, 3 a 5, e 5 a 7, para excelente, bom e regular, respectivamente. Observando as Tabelas~\ref{tab:accm_android} e~\ref{tab:accm_apps} percebe-se que os valores obtidos neste trabalho estão dentro do intervalo excelente ou bom, para os percentis 75 e 90, e dentro de bom ou regular para o percentil 95, q representa valores menos frequentes. No geral, os resultados indicam que o sistema tem uma boa complexidade ciclomática e que os aplicativos desenvolvidos para o mesmo acompanham essa mesma linha. \citeonline{meirelles2013} definiu intervalos semelhantes para códigos em C, e valores um pouco reduzidos para códigos em C++ e Java (0 a 2, 2 a 4, e 4 a 6 para os percentis 75, 90 e 95 respectivamente).  Os resultados encontrados para a API do sistema Android se encontram todos dentro desses intervalos, confirmando como um bom resultado. Já os aplicativos tem algumas exceções que extrapolam levemente esses valores, mas continuam em sua maioria dentro desses limites.

Baseando-se nessas observações, são considerados os seguintes intervalos:

\begin{itemize}
\item Valores abaixo 2 se mostraram muito frequentes para os aplicativos e para a API, e até 2.5 são considerados excelentes. É importante relembrar que uma complexidade ciclomática 2 implica em afirmar que 2 testes unitários resultam em 100\% de cobertura para esse trecho de código;
\item ACCM menor ou igual a 4 pode ser vista em todas as versões do Android e na grande maioria dos aplicativos dentro do percentil 90, sendo uma referencia para um valor maior mas ainda considerado bom. Os aplicativos do sistema quase não alcançaram esse valor;
\item Valores acima de 4 são considerados regulares e são pouco frequentes em ambos os casos, porém para a API do sistema o percentil 95 chegou a 6. Valores acima de 6 são bem raros e correspondem a uma quantidade estatisticamente desprezível para esta análise;
\end{itemize}

\subsection{Response For a Class}

Response for a Class é uma métrica que conta o número de métodos que podem ser executados a partir de uma mensagem enviada a um objeto dessa classe. O valor então é calculado pelo somatório de todos os métodos daquela classe, e todos os métodos chamados diretamente por essa classe. Uma classe com alto valor de RFC pode ser uma classe com um número muito grande de métodos, e/ou uma classe bastante dependente de outra(s) classe(s). Um valor Alto de RFC então pode indicar baixa coesão (LCOM alto) e alto acoplamento (CBO alto). 

O ideal em uma classe é obter métodos pequenos com tarefas atômicas e bem definidas, que correspondam às responsabilidades dessa classe. Essa métrica está diretamente relacionada a Number Of Methods (NOM), uma vez que um aumento neste ultimo implica necessariamente em um aumento em RFC. Uma classe que tenha alto RFC e muitos métodos (valor alto de NOM) pode indicar que está fazendo mais tarefas do que é sua responsabilidade fazer, necessitando talvez rever a sua implementação para aumentar sua coesão. Da mesma forma, um valor alto de RFC e valor baixo de NOM indica que uma classe está fazendo muito o uso de métodos de terceiros, podendo-se inferir que alguns métodos possam ser extraídos para essas classes que estão sendo tanto chamadas, com o objetivo de diminuir o acoplamento entre essas classes.

\begin{table}[!htb]
\scalefont{.7}
\documentclass[a4paper]{article}
\usepackage[T1]{fontenc}
\usepackage[latin1]{inputenc}
\begin{document}
\begin{tabular}{|l|l|l|l|l|l|l|l|l|l|l|l|}
\hline
version&min&1\%&5\%&10\%&25\%&50\%&75\%&90\%&95\%&99\%&max\\
\hline
android-1.6\_r1.2&0&0&0&0&2&10&31&79&133.8&357.480000000003&2858\\
\hline
android-1.6\_r1.5&0&0&0&0&2&10&31&79&133.8&357.480000000003&2858\\
\hline
android-2.0\_r1&0&0&0&0&2&10&31&79&131.5&350&2902\\
\hline
android-2.1\_r2.1p2&0&0&0&0&2&10&32&79&133.049999999999&352.869999999999&2923\\
\hline
android-2.2\_r1&0&0&0&0&2&9&30&77.9000000000005&131.45&372.449999999999&2754\\
\hline
android-2.2.3\_r2&0&0&0&0&2&9&30&78&131.15&372.150000000001&2754\\
\hline
android-2.3\_r1&0&0&0&0&1&8&27&76&129&358&2347\\
\hline
android-2.3.7\_r1&0&0&0&0&1&8&27&76&130&354.219999999999&2347\\
\hline
android-4.0.1\_r1&0&0&0&0&1&7&28&82&140&388&2871\\
\hline
android-4.0.4\_r2.1&0&0&0&0&1&7&28&81&141&391&2921\\
\hline
\end{tabular}
\end{document}

\scalefont{.7}
\caption{Percentis para a métrica \textit{Response For a Class} no Android}
\label{tab:rfc_android}
\end{table}

A API do sistema Android tende a ter um valor relativamente alto de RFC devido a forma como sua arquitetura foi desenhada, como pode ser visto na Tabela~\ref{tab:rfc_android}. Serviços do sistema são acessados muitas vezes através de objetos do sistema, e para seu uso correto alguns métodos devem ser chamados explicitamente. Por exemplo, para acessar o \textit{bluetooth}, não se chama diretamente um método de uma classe \textit{BluetoothAdapter}, pois os serviços do sistema geralmente estão encapsulados e são retornados por um método \textit{get()}, seguidos dos métodos que se deseja utilizar desse serviço. Por exemplo, para verificar dispositivos \textit{bluetooth} próximos, deve-se obter o \textit{adapter} via chamada estática de método para a própria classe para obter a instância, seguida de uma chamada de método para início de \textit{discovery} de dispositivos, e em seguida utilizar os métodos \textit{isDiscoverying()} e \textit{cancelDiscovery()} para controlar a busca. Um acesso direto a uma variável booleana removeria a necessidade da chamada de método \textit{isDiscoverying()}, entretanto perderia seu encapsulamento. De forma geral, encapsulamento de variáveis tende a aumentar o valor de RFC, que conta apenas métodos. 

Contribuindo para o aumento o resultado da busca de dispositivos bluetooth é realizado de forma assíncrona na forma de mensagens utilizando \textit{intents} (vide Capítulo~\ref{cap:android-os}), então mais um método é criado dentro de um \textit{receiver} (que pode ser a própria classe estendendo \textit{BroadcastReceiver}) para receber essa mensagem, aumentando um pouco o valor de RFC. Uma comunicação síncrona hipotética com uma chamada estática direta como \textit{BluetoothAdapter.discoverNearDevices()} retornando uma lista seria em teoria uma forma muito mais simples de ser utilizada, porém perderia a proteção do encapsulamento e deixaria de utilizar o comportamento em escopo de objeto para usar em escopo de classe, e também se perderia o maior controle sobre a própria busca que a API dá ao usuário com os métodos adicionais. Além disso, o encapsulamento de serviço dos sistema é um controle adicional que permite que o mesmo escalone melhor a utilização de recursos que necessitem de exclusão mútua. Por exemplo, um acesso direto a câmera dificultaria o sistema de dar acesso a 1 cliente de cada vez, pois afinal, o usuário não consegue usar a câmera, por exemplo, em dois aplicativos simultaneamente.

Comunicações assíncronas são muito usadas ao longo de todo o sistema para utilização de recursos, e então é necessário ter uma forma de receber mensagens de aplicativos e do sistema, o que é feito com a classe \textit{BroadcastReceiver} e implementando métodos específicos da mesma. Mesmo fora do contexto Android, comunicações assíncronas tendem a criar métodos adicionais de comunicação, como é visto no padrão \textit{Observer}, que se assemelha muito a essa comunicação por \textit{Intents}.

\begin{table}[!htb]
\scalefont{.7}
\begin{tabular}{|l|l|l|l|l|l|l|l|l|l|l|l|}
\hline
app&min&1\%&5\%&10\%&25\%&50\%&75\%&90\%&95\%&99\%&max\\
\hline
Launcher2&0&0&1&1&3.75&12.5&37.25&84.4&178.75&850.719999999999&1061\\
\hline
Settings&0&0&0&0&3&10&31&69&112&229.2&596\\
\hline
Camera2&0&0&0&0&1&6&22&65&113&352.199999999998&752\\
\hline
Bluetooth&0&0&0&2&8&25&68.75&131.3&205.6&468.34&658\\
\hline
VoiceDialer&2&2&2&2&6&14&37&59.4&87.8999999999999&137.58&150\\
\hline
QuickSearchBox&0&0&1&1&3&8&19&30&57.3&115.12&213\\
\hline
Calculator&1&1&1&1&6&8&13&42.8&62.4&78.08&82\\
\hline
Mms&0&0&0&0&2&10&30.5&76.6&122&247.36&788\\
\hline
ManagedProvisioning&1&1&1.85&2&2.25&7.5&49.5&66.4&80.2&93.3&97\\
\hline
SoundRecorder&2&2.84&6.2&10.4&23&28&94&130.6&142.8&152.56&155\\
\hline
Terminal&0&0.15&0.75&1&3&13&47.5&52.5&56.5&64.9&67\\
\hline
PackageInstaller&0&0.17&0.85&2.4&3&10.5&26&62.5000000000001&128&155.2&162\\
\hline
SpareParts&1&1.6&4&7.00000000000001&16&31&46&55&58&60.4&61\\
\hline
Tag&1&1.14&1.7&2.4&4&9&16&20.2&25.5&33.9&36\\
\hline
CertInstaller&1&1&1&1&3.5&6&14.5&86&109.5&128.3&133\\
\hline
KeyChain&0&0.17&0.85&1.7&4&9.5&13.75&21.1&26.3&27.66&28\\
\hline
Dialer&0&0&0&1&3&8.5&25&66.4&113.1&250.74&321\\
\hline
Browser&0&0&1&1&3.25&13&36.75&78.6&124.35&328.89&795\\
\hline
PhoneCommon&0&0.17&0.85&1&3&5.5&9&30.8&43.95&52.79&55\\
\hline
InCallUI&0&0&0&0&1&8&28.25&82.5&120.25&297.349999999999&434\\
\hline
CellBroadcastReceiver&1&1&1&2&5&15.5&26&33.4&54.8&75.63&84\\
\hline
OneTimeInitializer&2&2.18&2.9&3.8&6.5&11&15.5&18.2&19.1&19.82&20\\
\hline
LegacyCamera&0&0&0&0.200000000000003&3&11&38&77.6&139.6&400.76&742\\
\hline
Gallery2&0&0&0&0&3&12&33.75&77.7&110&312.14&595\\
\hline
BasicSmsReceiver&7&7.03&7.15&7.3&7.75&12.5&24.75&38.7&43.35&47.07&48\\
\hline
MusicFX&1&1&1&1.5&2.75&5.5&35.25&125&147.5&194.3&206\\
\hline
TvSettings&0&0&1&1&3&14&41&76.9&107.65&191.599999999999&775\\
\hline
Stk&1&1.16&1.8&2.6&6&18&45&125.6&179.2&304.64&336\\
\hline
UnifiedEmail&0&0&0&1&3&9&25&59&114.5&356.8&1012\\
\hline
Launcher3&0&0&0&0&2&9&31&77.8&144.6&515.640000000001&1407\\
\hline
Music&0&0&0&1&2&6.5&19.75&45.6000000000001&94.9999999999999&155.5&192\\
\hline
Camera&0&0&0&1&2&10&33&80&128.5&307.27&921\\
\hline
Email&0&0&0&1&2.5&9&28&60&93.1999999999999&192.22&399\\
\hline
Nfc&0&0&0&1&3&16&37&93.8&155&263.44&306\\
\hline
Gallery&0&0.87&1&1&4&17&31.5&68.3&118.7&224.66&296\\
\hline
ContactsCommon&0&0&0&1&3&9&22&60&105.5&199.4&271\\
\hline
Contacts&0&0&0&1&3&8.5&23&59&96.25&253.46&463\\
\hline
DeskClock&0&0&0&1&4.75&21&50.25&121.3&151.35&230.25&691\\
\hline
FMRadio&0&0&0&0&0&0.5&18.25&75.4&105.85&277.05&414\\
\hline
\end{tabular}
\newline
\begin{tabular}{|l|l|l|l|l|l|l|l|l|l|l|l|}
\hline
HTMLViewer&1&1.06&1.3&1.6&2.5&4&4.5&4.8&4.9&4.98&5\\
\hline
Calendar&0&0&0&1&4&13&37.5&109.6&160&422.799999999998&1291\\
\hline
Exchange&0&0&0&1&4&14&37.75&72.1&107.55&162.05&224\\
\hline
\end{tabular}

\scalefont{.7}
\caption{Percentis para a métrica \textit{Response For a Class} nos aplicativos nativos}
\label{tab:rfc_apps}
\end{table}

Alto valor de profundidade de árvore de herança em uma classe pode auxiliar no aumento da métrica RFC, uma vez que todos os comportamentos são herdados. Entretanto não é uma correlação direta significativa entre as duas, visto que a profundidade de herança raramente é alta para o sistema Android, como será discutido nas seções seguintes. Embora valor alto de DIT possa significar maior valor de RFC, valores muito altos de RFC tendem a indicar mais a falta de coesão e alto acoplamento, aumentando assim a complexidade estrutural da classe, e não uma profundidade de herança preocupante.
%TODO rever esse parágrafo que compara com DIT

Em suma, na API do sistema, o valor de RFC pode ser considerado alto, porém justificável. Inclusive, o acoplamento entre a própria API de desenvolvimento e o aplicativos faz com que o valor dessa métrica seja alto também para aplicativos desenvolvidos para o Android. E importante lembrar que componentes do sistema se comunicam da mesma forma com outros componentes do sistema como se comunicam com aplicativos desenvolvidos para o mesmo. A Tabela~\ref{tab:rfc_apps} demonstra os valores de RFC para aplicativos nativos.

\citeonline{meirelles2013} define como bons intervalos para projetos Java valores de 0 a 9, 10 a 26, e 27 a 59, para os percentis 75, 90 e 95, respectivamente. Pode-ses perceber que os valores na análise da API obtidos neste trabalho estão bem acima desse valor, estando em seu percentil 75 um valor perto de 30, que seria no máximo regular nessa escala.

Baseando-se em todas essas observações, são considerados os seguintes intervalos:

\begin{itemize}
\item Valores abaixo 31 se mostraram muito frequentes para API Android. Para os aplicativos do sistema, existe uma grande variância de valores, porém estando em sua grande maioria abaixo de 38 para o percentil 75.
\item RFC chegou a 130 em aplicativos nativos, porém no geral não alcançam o valor 85. Esse mesmo valor é o limite para a API do sistema.
\item Valores acima de 85 são valores considerados muito altos para a métrica RFC, e são pouco frequentes nos dados analisados.
\end{itemize}

Um intervalo de valores até 38 pode ser considerado bom para aplicativos, e regular no intervalo desse valor até o 85. Acima disso são considerados valores altos. Esses intervalos aqui encontrados estão mais próximos dos limites definidos por \citeonline{meirelles2013} para a Linguagem C do que para Java. Inclusive os valores aqui encontrados aqui para a API Android, essencialmente em Java, se mostram bastante semelhantes aos valores para o projeto Android como um todo, com predominância da linguagem C, encontrados por \citeonline{meirelles2013}. Essa semelhança pode levar a interpretação de que estilos semelhantes de estruturação e design são utilizados em todo o AOSP, independente da linguagem utilizada em cada módulo.
