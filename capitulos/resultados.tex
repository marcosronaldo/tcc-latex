\chapter{Resultados}
\label{cap:resultados}

%Neste capítulo serão discutidos os resultados da análise dos dados coletados, bem como resultados referentes ao modelo de regressão criado para utilizar os dados como um auxílio para desenvolvimento de projetos futuros.
%TODO criar explicação geral para o capítulo

\section{Análise preliminar}

Com os dados coletados e devidamente preparados, várias conclusões podem ser tiradas dos valores das métricas e sua evolução ao longo do tempo. Esta seção é focada na análise subjetiva dos dados, tentando explicar seu comportamento com relação às características do sistema, compará-los a outros estudos, analisá-los segundo a linguagem de programação utilizada, e até mesmo comparar com dados de métricas em aplicativos, utilizando os próprios aplicativos do sistema como base de comparação.

\subsection{Lines Of Code}
%TODO resumir aqui LOC e AMLOC
A primeira observação que deve ser feita quando analisando LOC nesse contexto é a diferenciação das linguagens. Embora um módulo em C seja mapeado para uma classe, arquivos fonte em C tendem a ser maiores que uma classe em Java, por exemplo, devido aos diferentes paradigmas que essas linguagem utilizam. Arquivos em C++ e Java também podem ter valores bem distintos para a mesma funcionalidade devido ao número de bibliotecas padrões que a linguagem apresenta e a natureza da própria sintaxe da linguagem. Dessa forma, comparações dessa métrica devem ser feitas somente dentro da mesma linguagem. Neste trabalho não serão feitas comparações diretas dos valores desta métrica, então ela será utilizada principalmente para relativização da comparação de outras métricas, uma vez que os valores de algumas delas podem ser relacionados a esta.

%TODO linkar tabela com percentils de LOC (SEPARAR VALORES PARA CÓDIGOS EM C e criar tabela separada)

%TODO mostrar gráfico com evolução da métrica para todas as linguagens em conjunto

\subsection{ACCM} %TODO colocar nomes por extenso
\subsection{CBO}
