\chapter{Resultados}
\label{cap:resultados}

%Neste capítulo serão discutidos os resultados da análise dos dados coletados, bem como resultados referentes ao modelo de regressão criado para utilizar os dados como um auxílio para desenvolvimento de projetos futuros.
%TODO criar explicação geral para o capítulo

\section{Análise preliminar}

Com os dados coletados e devidamente preparados, várias conclusões podem ser tiradas dos valores das métricas e sua evolução ao longo do tempo. Esta seção é focada na análise subjetiva dos dados, tentando explicar seu comportamento com relação às características do sistema, compará-los a outros estudos, e até mesmo comparar com dados de métricas em aplicativos, utilizando os próprios aplicativos do sistema como base de comparação.

Embora seja relevante comentar as diferenciações entre as linguagens e seus paradigmas para algumas métricas,  não há necessidade de separação entre os valores para linguagem C, procedural, e linguagens Java/C++, orientadas a objetos, uma vez que, dadas as proporções das mesmas apresentadas nos dados, não há relevância estatística para tal. Entretanto em algumas métricas algumas observações teóricas possam ser ressaltadas, embora, mais uma vez, não haja implicação substancial no resultados gerais apresentados.

\subsection{Average Method Lines Of Code}
%TODO resumir aqui LOC e AMLOC
A primeira observação que deve ser feita quando analisando LOC nesse contexto é a diferenciação das linguagens. Embora um módulo em C seja mapeado para uma classe, arquivos fonte em C tendem a ser maiores que uma classe em Java, por exemplo, devido aos diferentes paradigmas que essas linguagem utilizam. Arquivos em C++ e Java também podem ter valores bem distintos para a mesma funcionalidade devido ao número de bibliotecas padrões que a linguagem apresenta e a natureza da própria sintaxe da linguagem. Dessa forma, comparações dessa métrica devem ser feitas somente dentro da mesma linguagem. Neste trabalho não serão feitas comparações diretas dos valores desta métrica, então ela será utilizada principalmente para relativização da comparação de outras métricas, quando aplicável, uma vez que os valores de algumas delas podem ser relacionados a esta. 

A métrica LOC por si só não será discutida aqui, pois seu valor é mais relativo e deve ser comparado com outras métricas para ter significado mais completo. Uma classe com valor alto de LOC pode ter um baixo valor de AMLOC e valor maior para NOM, ainda mantendo um valor aceitável de LCOM. Em suma, a análise de outras métricas abrange as explicações relacionadas a métrica LOC e também a NOM, então essas métricas de tamanho não serão explanadas em separado, mas juntamente com a explicação de outras métricas.

Valores baixos de AMLOC são sempre preferíveis pois métodos mais enxutos tem menor responsabilidade, portanto estão mais sujeitos a reuso, e também são mais fáceis de se ler e se modificar. Entretanto essa métrica é preciso ser analisada em conjunto com outras métricas, como LCOM e RFC. Uma classe com muitos métodos privados pequenos tende a ter um valor maior de RFC, o que não implica que esteja mal projetada, desde que os métodos ali presentes estejam bem posicionados segundo o padrão de projeto OO especialista da informação, mantendo por consequencia um baixo valor de LCOM. Como referência geral de resultados para AMLOC, quanto menor o valor, melhor o resultado.

\begin{table}[!htb]
\scalefont{.7}
\documentclass[a4paper]{article}
\usepackage[T1]{fontenc}
\usepackage[latin1]{inputenc}
\begin{document}
\begin{tabular}{|l|l|l|l|l|l|l|l|l|l|l|l|}
\hline
version&min&1\%&5\%&10\%&25\%&50\%&75\%&90\%&95\%&99\%&max\\
\hline
android-1.6\_r1.2&0&0&0&0&2.33333333333333&5.57142857142857&11.5&21.5&30&65.8742857142858&312\\
\hline
android-1.6\_r1.5&0&0&0&0&2.33333333333333&5.57142857142857&11.5&21.5&30&65.8742857142858&312\\
\hline
android-2.0\_r1&0&0&0&0&2&5.55555555555556&11.5&21.8461538461538&30.1916666666666&67.8124999999999&390.5\\
\hline
android-2.1\_r2.1p2&0&0&0&0&2&5.625&11.5&21.8571428571429&30.3366666666666&68.4017499999999&395\\
\hline
android-2.2\_r1&0&0&0&0&1.75892857142857&5.8&12.8&26.5&44.2135714285714&156.66&1034\\
\hline
android-2.2.3\_r2&0&0&0&0&1.76587301587301&5.81666666666666&12.8214285714286&26.5&44.1664285714285&156.62&1034\\
\hline
android-2.3\_r1&0&0&0&0&1&5.8&13.6&30.1787878787879&55.3636363636363&164.773333333334&1034\\
\hline
android-2.3.7\_r1&0&0&0&0&1&5.83333333333333&13.7064950980392&30&54.0590277777778&163.397272727272&1034\\
\hline
android-4.0.1\_r1&0&0&0&0&1&5.85714285714286&14&31&54.3731481481482&162.418181818181&1034\\
\hline
android-4.0.4\_r2.1&0&0&0&0&1&5.85714285714286&14&31&53.9791666666667&162&1034\\
\hline
\end{tabular}
\end{document}

\scalefont{.7}
\caption{Percentis para a métrica \textit{Average Method Lines of Code} no Android}
\label{tab:amloc_android}
\end{table}

A Tabela~\ref{tab:amloc_android} apresenta os valores para a métrica AMLOC nas versões do Android analisadas. É facilmente perceptível que a média de linhas de código por método não teve variação relevante. Em todas as versões analisadas, os valores muito frequentes, isto é, percentil 75, são métodos com até 14 linhas de código, enquanto de 14 a 30 aparecem como frequentes, e 31 a 55 pouco frequentes. Esses são valores que estão de acordo com os apresentados em \citeonline{meirelles2013}, porém levemente menores para os percentis 75 e 90, com aproximadamente 3 linhas de código a menos por método. É possível perceber que os valores se mostraram bem semelhantes para o projeto Android, mesmo considerando o fato que este trabalho estuda apenas a API de desenvolvimento de aplicativos, essencialmente em java e dentro do diretório ``\textit{frameworks}'' do AOSP, e \citeonline{meirelles2013} analiza todo o código fonte do sistema, que apresenta em sua totalidade uma maior proporção da linguagem C em relação as demais. Esses valores são subsídios para reafirmar que arquivos em C em geral, tem uma maior utilização de linhas de código do que arquivos em Java. \citeonline{oliveira2013} comenta que as diferenças entre as linguagems C/C++/Java para esta métrica não é significativa, uma vez que a sintaxe entre as 3 é bastante semelhante. Dada essa afirmação, podemos comparar os intervalos definidos por ele, chegando a conclusão de que os valores das métricas estão, para todas as versões, abaixo dos valores e regular para os percentis 75 e 90, o que é um bom resultado.

\begin{table}[!htb]
\scalefont{.7}
\begin{tabular}{|l|l|l|l|l|l|l|l|l|l|l|l|l|}
\hline
app&classes&min&1\%&5\%&10\%&25\%&50\%&75\%&90\%&95\%&99\%&max\\
\hline
Launcher2&161&0&0&1&1&3&6.33&10.78&16.9&24.38&48.36&57.5\\
\hline
Settings&722&0&0&0&0&3.63&8&15&21.47&28.5&49.4&80.42\\
\hline
Camera2&462&0&0&0&0&1&4.5&9.85&16.09&21.5&38.28&66.67\\
\hline
Bluetooth&239&0&0&0&2.83&6.02&10.79&22.16&39.86&59.84&111.11&221\\
\hline
QuickSearchBox&196&0&0&1&1&3&4.39&6.17&10.53&13.15&22.28&32\\
\hline
Calculator&10&1&1&1&1&6.67&8.92&13.33&23.26&27.38&30.68&31.5\\
\hline
Terminal&17&0&0.15&0.75&1.85&3.09&8.12&15.29&20.92&29.38&48.27&53\\
\hline
PackageInstaller&19&0&0.68&3.4&4&4.63&6.59&16.98&18.9&22.89&31.45&33.59\\
\hline
Dialer&215&0&0&0&1&3&7&11.13&16.89&19.98&32.02&61.33\\
\hline
Browser&259&0&0&1&1&3.5&6.89&11&19&25.95&46.02&55.33\\
\hline
InCallUI&117&0&0&0&0&1&4.23&12&18.75&23.32&40.77&58\\
\hline
LegacyCamera&214&0&0&0&0.2&4&8.64&15.78&25.47&32.18&69.42&112.67\\
\hline
Gallery2&895&0&0&0&0&3&6&11.5&17.38&21.67&44.36&107\\
\hline
BasicSmsReceiver&5&8.67&8.83&9.47&10.27&12.67&16.33&18.75&18.9&18.95&18.99&19\\
\hline
UnifiedEmail&872&0&0&0&1&3&5&9.71&17&23.67&37.95&139.63\\
\hline
Launcher3&354&0&0&0&0&2.73&5.13&10.59&17.17&24.79&54.71&163.5\\
\hline
Music&75&0&0&0&1&4.1&9.51&16.89&21.76&28.0&48.32&90\\
\hline
Camera&253&0&0&0&1&3&7.42&13&22.37&31.45&72.23&112.67\\
\hline
Email&400&0&0&0&1&3.58&8&15.35&24.49&31.61&63.28&128\\
\hline
Nfc&178&0&0&0&1&3&9.64&18.5&31.63&38&42.48&70.5\\
\hline
Gallery&89&0&0.87&1&1&4&7.63&12.67&19.0&28.6&53.12&55\\
\hline
ContactsCommon&292&0&0&0&1&3.23&7.1&13&19&23.88&34.5&53.33\\
\hline
Contacts&265&0&0&0&1&3&6.45&11.5&18.61&23.72&63.53&86\\
\hline
DeskClock&121&0&0&0&1&5&9.16&15.26&24.02&27.3&30.71&40.13\\
\hline
HTMLViewer&4&5&5.12&5.6&6.2&8&11&14.5&16.6&17.3&17.86&18\\
\hline
Calendar&216&0&0&0&1&5&11.67&19.58&30.95&39.3&90&115.5\\
\hline
Exchange&135&0&0&0&1&4&10.01&17.31&28.41&34.65&44.41&51.25\\
\hline
\end{tabular}

\scalefont{.7}
\caption{Percentis para a métrica \textit{Average Method Lines of Code} nos aplicativos nativos}
\label{tab:amloc_apps}
\end{table}

Os valores apresentados na análise são relativamente baixos quando comparados com outros softwares livres, como demonstrado por \citeonline{meirelles2013}. Da mesma forma, quando olhamos os valores aplicativos do sistema, demonstrados na Tabela~\ref{tab:amloc_apps}, podemos perceber uma grande semelhança nos resultados. Embora alguns poucos aplicativos tenham valores mais elevados para essa métrica, pode-se perceber que os intervalos se mantém válidos para a grande maioria dos aplicativos. Esses valores de aplicativos foram retirados dos aplicativos nativos da ultima versão do sistema analisada (Lollipop 5.1.0), e continuam se mantendo semelhantes ao sistema, como o próprio acoplamento à API sugere.

Em linhas gerais, os aplicativos do sistema também se mantém dentro dos intervalos bom e regular definidos em \citeonline{oliveira2013}. Os valores para o percentil 95 também se encontram abaixo do valor regular, na maioria dos casos.

Em suma, os valores para os aplicativos se assemelham muito com os valores para as versões da API Android analisadas, levando então a conclusão de que os mesmos intervalos são válidos para as métricas em ambos os casos, embora se possa esperar valores menores em aplicativos, porem com uma maior variância. Essa variância se dá pelo diferente propósito de cada aplicativo, que utiliza pedaços variados do sistema e tem sua codificação adaptada para seu propósito.

Intervalos encontrados:

\begin{itemize}
\item Valores abaixo de 14 se mostraram muito frequentes para os aplicativos e para a API;
\item Enquanto no sistema os valores para o percentil 90 se encontram abaixo de 31, nos aplicativos eles alcançam em poucos casos, ficando em sua maioria abaixo de 25;
\item Valores acima de 31 são pouco frequentes em ambos os casos;
\end{itemize}

\subsection{ACCM} %TODO colocar nomes por extenso
\subsection{CBO}
