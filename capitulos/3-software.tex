\chapter{Engenharia de Software}
\label{cap:software}

Engenharia de software é definida pela \citeonline{swebok} como aplicação de uma abordagem sistemática, disciplinada e mensurável ao desenvolvimento, operação e mautenção de software. \citeonline{swebok} é um documento bastante consolidado que reúne diversos conceitos relacionados a engenharia de software, e foi um marco para o reconhecimento da engenharia de software como engenharia de fato. Embora sem muitos detalhes, as seções seguintes citam alguns dos processos da engenharia de software que são importantíssimos em todo o ciclo de vida do produto de software, que inclui desde a concepção do produto até a manutenção do mesmo em ambiente de produção.

\section{Requisitos}

Área de conhecimento em requisitos de software é responsável pela elicitação, analise, especificação e validação de requisitos de software, bem como a manutenção gerenciamento desses requisitos durante todo o ciclo de vida do produto \cite{swebok}.

Requisitos de software representam as necessidades de um produto, condições que ele deve cumprir para resolver um problema do mundo real que o software pretende atacar. Essas necessidades podem ser funcionalidades que o software deve apresentar para o usuário, chamados requisitos funcionais, ou outras condições que restringem as funcionalidades de alguma forma, seja por exemplo sobre tempo de execução, requisitos de segurança ou outras restrições, conhecidas como requisitos não funcionais.

Requisitos de software geralmente tem como fonte o próprio cliente que contrata o serviço do desenvolvimento de sofwate, e são estraídos da descrição de como esse cliente vê o uso do sistema. Todo esse processo é muitas vezes chamado de "engenharia de requisitos".

Requisitos de software podem ser representados de diversas formas, sendo possível a utilização de vários modelos distintos. Na metodologia ágil scrum, por exemplo, os requistos normalmente são registrados na forma de User Stories, onde são geralmente descritos na visão do usuário do sistema. Em outros contextos, podem ser descritos em casos de uso, com descrições e diagramas, ou mesmo outras formas de representação.

\section{Desenho de software}

Desenho de software é o processo de definição da arquitetura, componentes, interfaces, e outras características de um sistema ou um componente \cite{swebok}. É durante o desenho de software que os requisitos são traduzidos na estrutura que dará base ao software sendo desenvolvido. As necessidades então são traduzidas em modelos, que descrevem os componentes e as interfaces entre os componentes. A partir desses modelos é possível avaliar a validade da solução desenhada e as restrições associadas a mesma, sendo possível avaliar diferentes soluções antes da implementação do software. A partir do design da arquitetura, é possível prever se alguns requisitos elicitados podem ou não ser atingidos com determinada solução, e mudá-la conforme necessário com pouco ou mesmo nenhum custo adicional.

A área de desenho de software pode variar conforme a tecnologia sendo utilizada. A arquitetura do sistema pode variar conforme o sistema operacional alvo, ou mesmo conforme a linguagem de programação que se está utilizando no desenvolvimento. Existem vários princípios de design de software amplamente conhecidos que se aplicam a uma imensidade de situações, sempre com o intuito de encontrar a melhor solução para cada situação e deixar o software modularizado e manutenível.

\section{Construção de software}

Essa área de conhecimento é responsável pela codificação do software, por transformar a arquitetura desenhada e seus modelos em código fonte. A construção de software está extremamente ligada ao desenho e às atividades de teste, partindo do primeiro e gerando insumo para o segundo \cite{swebok}.

Várias medidas podem ser coletadas do próprio código pra auxiliar a avaliação da qualidade do produto sendo construído e gerar insumo para o próprio desenvolvedor reavaliar sua implementação antes da fase de testes.

\section{Testes}

Testes de software consistem em verificar se o produto de software se comporta da forma esperada em um determinado conjunto de casos específicos, selecionados com o intuito de representar o maior número de situações diferentes que podem ocorrer durante o uso do sistema, com o software em execução. Os testes têm que ser projetados para checar se o software está de acordo com as necessidades do usuário, procedimento conhecido como validação, e para verificar se as funcionalidades estão de acordo com a especificação, procedimento conhecido como verificação \cite{swebok}. Testes podem ser realzados em vários níveis, desde o teste de pequenos trechos de código até a interação entre componentes e o teste da interface gráfica do usuário.

Existem vários tipos de testes aplicáveis a determinados tipos de sistema. De acordo com as necessidades e o ambiente onde o sistema irá funcionar, vários testes podem ser ou não necessários para garantir o funcionamento do sistema sob diversas condições. Sistemas web podem exigir testes de carga e stress para avaliar a quantidade de usuários simultaneos suportados, por exemplo. Sistemas críticos já também necessitam de testes de recuperação, para avaliar a capacidade do sistema de restaurar seu funcionamento após algum tipo de falta.

Ter uma boa suite de testes é de grande valia para a manutenção de um produto de software, uma vez que sempre que uma modificação precisar ser feita no sistema, é possível verificar de forma automatizada se algum comportamento foi indevidamente alterado pela modificação realizada.

\section{Manutenção}

A manutenção de software trata dos esforços de desenvolvimento com o software já em produção, isto é, em funcionamento no seu devdo ambiente. Problemas que passaram despercebidos durante as fases de construção e testes são encontrados e corrigidos durante a manutenção do sistema. Da mesma forma, o usuário pode requisitar novas funcionalidades que ele não havia pensado antes do uso do sistema, e o desenvolvimento dessas novas funcionalidades é também tratado como manutenção uma vez que o software já se encontra em produção, num processo conhecido como evolução de software. 

Revisões de código e esforços com manutenibilidade também podem ser consideradas atividades de manutenção, embora possam acontecer antes do sistema entrar em produção.

A manutenção de software acontece por período mais longo que as demais fases do desenvolvimento do software citadas nos tópicos anteriores, ocupando a maior parte do ciclo de vida do produto.

\chapter{ESW e Android}
\label{cap:eswandroid}

%corrigir citações

\section{Arquitetura}

Android provê um framework para desenvolver aplicativos baseado nos componentes descritos no capítulo 1 (citar aqui). Os aplicativos são construídos com qualquer combinação desses componentes, que podem ser utilizados individualmente, sem a presença dos demais. Cada um dos componentes pode ser uma entrada para o aplicativo sendo desenvolvido.

A comunicação entre os componentes de cada aplicativo é feita por meio de Intents. Como descrito no capítulo 1 (citar aqui), Intents são geralmente recebidos por receivers que estão registrados para recebê-los. Entretanto, activities e services também utilizam desse mecanismo para ser iniciados e finalizados. Quando um Intent é enviado ao sistema via broadcast, ele é recebido pelo sistema, que seleciona o melhor componente para tratá-lo, e então inicia o componente que o recebeu independente da aplicação a que ele pertença. Assim como já descrito, permissões podem ser criadas para restringir essa comunicação global, mas a principio qualquer componente pode receber um intent de qualquer outra. 

De forma geral, para desenhar uma arquitetura para sistema android deve-se levar em conta todos esses componentes e conhecer bem sua aplicabilidade. Se for necessário ter algum processo em background independente de feedback do usuário, por exemplo, deve-se utilizar do componente service. Caso contrário, quando o usuário fechar a interface gráfica do aplicativo, o aplicativo terá sua execução pausada e seu estado salvo para retorno posterior, deixando de executar alguma tarefa que não deveria ter sido interrompida. Como um exemplo mais paupável, o aplicativo do facebook precisa necessariamente utilizar de um serviço para que, mesmo quando não estiver em execução, notificações de mensagens e atualizações de status cheguem na tela do usuário.

Embora pareça restringir o design de aplicativos, o uso desses componentes unifica a forma com que os aplicativos são desenvolvidos. A API do android já disponibiliza acesso a vários recursos de hardware do sistema na foram de componentes, e assim como é possível utilizar os componentes do sistema, é possível utilizar componentes de qualquer outra aplicação da mesma maneira - se a permissão for concedida- , e criar os seus próprios componentes para uso de terceiros de forma padronizada. Em suma, é tão fácil usar componentes criados por alguma aplicação qualquer quanto componentes internos do sistema graças ao framework de desenvolvimento android.