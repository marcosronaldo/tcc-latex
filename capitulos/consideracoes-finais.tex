\chapter{Conclusão}
\label{consideracoes-finais}

Este trabalho teve como objetivo principal o monitoramento de métricas estáticas de código fonte, essencialmente métricas OO, e fazer um estudo da evolução de seus valores nas diferentes versões da API do sistema operacional Android, estudar as semelhanças com aplicativos do sistema e então verificar a possibilidade de utilizar os dados obtidos para auxiliar no desenvolvimento de aplicativos para o Android.

Foram conceituadas e aplicadas várias métricas na API de desenvolvimento de aplicativos Android a fim de avaliar a qualidade do código fonte da API em relação aos valores das métricas e sua distribuição dentro do sistema. Foi realizado um estudo da própria plataforma Android e sua arquitetura para auxiliar na interpretação de valores de métricas aplicados neste contexto. 

Foi verificado neste trabalho que a API e aplicativos desenvolvidos para o Android tem muitas semelhanças no que diz respeito a métricas OO, o que reforça a afirmação de alguns trabalhos aqui citados no sentido de existir um alto acoplamento entre eles. Os valores de métricas encontrados para os aplicativos estão muito semelhantes com os do sistema, refletindo um estilo arquitetural intrínseco da plataforma.

O cálculo de similaridade proposto tinha o objetivo principal de verificar se um aplicativo se aproximava da API Android, com o intuito de avaliar sua qualidade com esse referencial. O resultado mostrou que \textit{scores} negativos se mostram valores melhores que os da API, e \textit{scores} positivos são preocupantes. Assim como o objetivo propôs, o \textit{score} calculado pode ser utilizado como um indicador de problemas arquiteturais, caso supere o valor 0, que são os valores da API, em mais que algumas dezenas. Esse indicador de qualidade de código em comparação com a API parte da premissa de que a API tem uma boa arquitetura, e portanto terá sua validade enquanto essa premissa for verdadeira.

Em relação à questão de pesquisa e às hipóteses levantadas no Capítulo~\ref{cap:metodologia}, temos as seguintes observações:

\begin{itemize}
\item \textit{H1 - É possível identificar padrões e tendências na evolução da arquitetura do sistema Android e nos aplicativos desenvolvidos para ele.}
\end{itemize}

Temos a resposta de H1 na análise da evolução das métricas do código fonte contida no Capítulo~\ref{cap:analise_exploratoria}, onde foi verificado que é possível identificar um padrão de comportamento nos valores das métricas, que indica uma permanência dos valores de métricas OO em certos intervalos, independentemente da versão ou de seu tamanho. Da mesma forma, os aplicativos parecem obedecer um mesmo padrão que foi identificado para a API, acompanhando os valores na grande maioria dos casos.

\begin{itemize}
\item \textit{H2 - O desenvolvimento de aplicativos Android pode ser guiado pelo resultado de uma análise evolutiva do código do próprio sistema.}
\end{itemize}

H2 é uma hipótese que também é avaliada a partir do acoplamento entre a API e seus aplicativos. A validade dos mesmos intervalos de valores para ambos os casos implica na utilização dos mesmos como referência no desenvolvimento de aplicativos. A proposta de \textit{score} de similaridade é uma tentativa de utilização dos dados obtidos com esse propósito de auxiliar desenvolvedores na avaliação da qualidade do software em desenvolvimento, e como já comentado, esse cálculo de similaridade cumpriu seu objetivo.

\begin{itemize}
\item \textit{H3 - Uma grande aproximação ao sistema implica em uma boa qualidade de código.}
\end{itemize}

H3 se torna verdadeira a medida que encontramos excelentes valores de métricas para a API Android na análise no Capítulo~\ref{cap:analise_exploratoria}. Se aproximar, em termos de métricas de código fonte, de uma arquitetura consolidada e que contém ótimos valores de métricas, sem dúvidas pode ser utilizado como um indicador de qualidade de produto de software para ser utilizado em aplicativos.

\begin{itemize}
\item \textit{H4 - As decisões arquiteturais teóricas aplicadas no estudo de caso e-lastic estão relacionadas com decisões arquiteturais baseadas em métricas.}
\end{itemize}

Foi observado que as decisões tomadas ao longo do desenvolvimento com base em padrões de projeto estão relacionadas às decisões tomadas com base em métricas, uma vez que decisões tomadas baseadas em métricas tentariam levar os valores para os menores possíveis, tentando alcançar valores em intervalos ótimos, como os que foram encontrados no aplicativo parcial ``e-lastic'' desenvolvido durante as etapas iniciais deste trabalho. Foram tomadas decisões arquiteturais com base em padrões de projeto e princípios de design levando em conta experiência de programação com a plataforma, e o \textit{score} encontrado utilizando o indicador de similaridade proposto reforça a ideia de que isso resultou em uma boa arquitetura.

\begin{itemize}
\item \textit{QP - É possível monitorar métricas estáticas de código fonte de aplicativos Android de acordo com a análise de intervalos e aproximação do código do sistema?}
\end{itemize}

Por fim, este trabalho mostrou que a semelhança entre os dois faz com que os valores de referência sejam muito parecidos e podem então ser unificados em um só intervalo de referência. Assim, os intervalos definidos que caracterizam o código fonte da arquitetura do sistema são aplicáveis também aos aplicativos. A própria evolução do sistema demonstrou um certo padrão de permanência dos valores independentemente do tamanho do projeto, uma vez que a API aumentou seu tamanho em quatro vezes ao longo das versões analisadas, mas não teve variação significativa no tamanho das métricas. Esses resultados mais uma vez implicam na validade dos intervalos para o escopo de aplicativos.

\section{Limitações}

Uma limitação encontrada ao longo do trabalho foi a quantidade de dados utilizados para análise. Poucas versões do sistema Android puderam ser analisadas devido ao tempo que cada análise leva para ser concluída. Dados de poucas versões dificultam a generalização dos resultados para toda a plataforma, além de dificultar detecção de padrões nos valores de métricas na evolução do sistema.

Outra possível limitação foram as métricas selecionadas. Embora tenham sido utilizadas métricas consolidadas para arquiteturas OO, um possível argumento contra este trabalho é que as métricas aqui trabalhadas podem não ser suficientes para avaliar a qualidade da arquitetura, visto que diversos estudos utilizam um conjunto diferente de métricas. Várias métricas não foram usadas devido ao escopo reduzido deste trabalho, além do que seria necessária a utilização de mais de uma ferramenta de coleta.

Um fator que pode ameaçar a validade de alguns resultados deste trabalho foi a forma de normalização dos dados para o cálculo de similaridade proposto no Capítulo~\ref{cap:analise_exploratoria}. Foi feita uma relativização dos valores com relação a uma referência frequente dentro da própria métrica, porém podem ser encontradas outras formas melhores de equivalência entre os valores das métricas.

A respeito do cálculo de similaridade, alguns valores aqui utilizados, como os pesos aplicados no cálculo, embora tenham algum fundamento matemático, são arbitrários e podem ser substituídos por um estudo mais detalhado e focado nesse propósito.

\section{Trabalhos Futuros}

Propomos inicialmente a coleta e análise de métricas para outras versões do Android. Com mais amostras, a análise pode ser feita de forma mais detalhada, além do que pode ser melhor estudada uma proposta de regressão de valores para algumas métricas.

A utilização de \textit{machine learning} (ML) poderia ser uma boa contribuição para detecção de padrões no sistema e auxílio na análise dos valores das métricas. Um método de ML que obtivesse os dados ideais da arquitetura seria de grande valia para comparação com os intervalos definidos.

Incluir mais métricas também é uma contribuição para este trabalho, avaliando a distribuição e evolução das mesmas ao longo das versões do sistema. Além disso, métricas adicionais podem melhorar o fator de similaridade que foi proposto.

Sobre o cálculo de similaridade, é interessante o testar com outras formas de normalização dos dados, e avaliar os resultados com valores distintos de normalização, e pesos distintos tanto para os percentis quanto para as métricas. Também pode ser criada uma escala de valores para melhor classificar o score de saída.

Além disso, mais informações poderiam ser apresentadas com o \textit{score} de similaridade, como por exemplo dicas de refatoração obtidas a partir das diferenças mais discrepantes que contribuíram para o valor de saída. Isso auxiliaria bastante desenvolvedores iniciantes que não tem conhecimento profundo sobre interpretação de valores de métricas de código fonte, pois eles poderiam receber diretamente um indicador de qualidade com dicas de melhoria, sem olhar diretamente os valores das métricas.