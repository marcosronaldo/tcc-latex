\chapter{Considerações finais}
\label{consideracoes-finais}

Este trabalho teve como objetivo principal o monitoramento de métricas estáticas de código fonte, essencialmente métricas OO, e fazer um estudo da evolução de seus valores nas diferentes versões da API do sistema operacional Android, estudar as semelhanças com aplicativos do sistema e então verificar a possibilidade de utilizar os dados obtidos para auxiliar no desenvolvimento de aplicativos para o Android.

Foram conceituadas e aplicadas várias métricas na API de desenvolvimento de aplicativos Android a fim de avaliar a qualidade do código fonte da API em relação aos valores das métricas e sua distribuição dentro do sistema. Foi realizado um estudo da própria plataforma Android e sua arquitetura para auxiliar na interpretação de valores de métricas aplicados neste contexto. 

Foi verificado neste trabalho que a API e aplicativos desenvolvidos para o Android tem muitas semelhanças no que diz respeito a métricas OO, o que reforça a afirmação de alguns trabalhos aqui citados no sentido de existir um alto acoplamento entre eles. Os valores de métricas encontrados para os aplicativos estão muito semelhantes com os do sistema, refletindo um estilo arquitetural intrínseco da plataforma.

O cálculo de similaridade proposto tinha o objetivo principal de verificar se um aplicativo se aproximava da API Android, com o intuito de avaliar sua qualidade com esse referencial. O resultado mostrou que\textit{scores} negativos se mostram valores melhores que os da API, e \textit{scores} positivos são preocupantes. Assim como o objetivo propôs, o \textit{score} calculado pode ser utilizado como um indicador de problemas arquiteturais, caso supere o valor 0, que são os valores da API, em mais que algumas dezenas. Esse indicador de qualidade de código em comparação com a API parte da premissa de que a API tem uma boa arquitetura, e portanto terá sua validade enquanto essa premissa for verdadeira.

\section{Conclusão}

Respondendo a pergunta QP1, que pergunta se é possível monitorar código fonte de aplicativos de acordo com o código do sistema, este trabalho mostrou que a semelhança entre os dois faz com que os valores de referência sejam muito parecidos e podem então ser unificados em um só intervalo de referência. Assim, os intervalos definidos que caracterizam o código fonte da arquitetura do sistema são aplicáveis também aos aplicativos. A própria evolução do sistema demonstrou um certo padrão de permanência dos valores independentemente do tamanho do projeto, uma vez que a API aumentou seu tamanho em quatro vezes ao longo das versões analisadas, mas não teve variação significativa no tamanho das métricas. Esses resultados mais uma vez implicam na validade dos intervalos para o escopo de aplicativos.

Em relação as hipóteses levantadas no Capítulo~\ref{cap:metodologia}, temos a resposta de H1 na análise da evolução das métricas do código fonte contida no Capítulo~\ref{cap:resultados}, onde foi verificado que é possível identificar um padrão de comportamento nos valores das métricas, que indica uma permanência dos valores de métricas OO em certos intervalos, independentemente da versão ou de seu tamanho. Da mesma forma, os aplicativos parecem obedecer um mesmo padrão que foi identificado para a API, acompanhando os valores na grande maioria dos casos.

H2 é uma hipótese que também é avaliada a partir do acoplamento entre a API e seus aplicativos, como foi identificado. A validade dos mesmos intervalos de valores para ambos os casos implica na utilização dos mesmos como referência no desenvolvimento de aplicativos. A proposta de \textit{score} de similaridade é uma tentativa de utilização dos dados obtidos com esse propósito de auxiliar desenvolvedores na avaliação da qualidade do software em desenvolvimento, e como já comentado, esse cálculo de similaridade cumpriu seu objetivo.

H3 se torna verdadeira a medida que encontramos excelentes valores de métricas para a API Android na análise no Capítulo~\ref{cap:resultados}. Se aproximar, em termos de métricas de código fonte, de uma arquitetura consolidada e que contém ótimos valores de métricas, sem dúvidas pode ser utilizado como um indicador de qualidade de produto de software.

Por fim, foi observado, para a hipótese H4, que as decisões tomadas ao longo do desenvolvimento do aplicativo com base em padrões de projeto estão relacionadas às decisões tomadas com base em métricas, uma vez que decisões tomadas baseadas em métricas tentariam levar os valores para os menores possíveis, tentando alcançar valores em intervalos ótimos, como os que foram encontrados no aplicativo parcial desenvolvido durante as etapas iniciais deste trabalho. Foram tomadas decisões arquiteturais com base em padrões de projeto e princípios de design levando em conta experiência de programação com a plataforma, e o \textit{score} encontrado utilizando o indicador de similaridade proposto reforça a idéia de que isso resultou em uma boa arquitetura.

\section{Trabalhos Futuros}
