\chapter{Estudo de Caso}

\section{E-Lastic}
O E-lastic é um sistema eletrônico que monitora e controla a execução de exercícios físicos realizados com equipamento que impõe sobrecarga à movimentação de segmentos corporais por meio de resistência elástica. Geralmente o controle de sobrecarga gerada por elementos elásticos é baseado na percepção subjetiva do esforço, com base na sensação de fadiga experimentada durante o exercício, e portanto o praticante não têm controle do esforço aplicado no exercício.

O produto em desenvolvimento apresenta um aparelho portátil, voltado para o controle de atividades físicas em ambientes fechados ou abertos. Trata-se de um sistema eletrônico embarcado que realiza o processamento digital do sinal originado num sensor de força e associa essas informações com variáveis de espaço e tempo, de forma a gerar informações suficientes para o controle e prescrição de exercícios resistidos. O sensor de força será fornecido calibrado juntamente com a central de processamento. Esse sistema eletrônico permite a acoplagem do implemento elástico para a realização do exercício a ser monitorado, e interfaceia com o usuário por meio de um aplicativo desenvolvido para um dispositivo móvel. De forma geral, durante o exercício físico, a força aplicada pelo usuário ao elemento elástico é calculada no microcontrolador e enviada juntamente com as demais informações via Bluetooth para um dispositivo móvel com o e-lastic app, que contém opções de controle para a realização do exercício físico.

Para a utilização do aparelho, o usuário deve selecionar o implemento elástico mais adequado ao exercício que será praticado, com resistência elástica compatível com o esforço que será aplicado. Um sensor de força compatível é acoplado ao implemento elástico e ligado ao módulo de processamento por uma entrada específica. O usuário então seleciona via aplicativo móvel um dos três modos de execução de exercício: dinâmico, isométrico e potência. Seleciona em seguida a configuração da sua rotina de treino, com os parâmetros de controle para o exercício específico. A partir daí o exercício pode ser iniciado, e informações precisas de controle são apresentadas na tela do dispositivo móvel em tempo real. Vibração do dispositivo, estímulos sonoros e outros tipos de sinais podem ser utilizados para dar biofeedback ao usuário sobre o exercício em curso. Após a realização do exercício, é apresentado um resumo da seção de treino executada. É possível que se apresente um histórico de treinamento dos indivíduos que realizam suas atividades utilizando o e-lastic.

Os parâmetros para controle podem variar de exercício para exercício, porém em todos eles será selecionado um intervalo de intensidade, isto é, carga máxima e mínima em quilogramas, em que o exercício será trabalhado. As opções de exercícios são espécies de rotinas pré programadas, modos de utilização da resistência elástica para o exercício físico:

\begin{enumerate}
\item Modo dinâmico - Realização de movimentos cíclicos em velocidade lenta e cadência controlada. Os parâmetros de controle para esse exercício são número de séries de exercícios, número de repetições por série e a duração do descanso entre as séries.
\item Modo isométrico - Contrações estáticas em que o usuário mantém uma posição e sustenta a sobrecarga elástica nessa posição por um determinado tempo. Semelhante ao modo dinâmico, no modo isométrico o usuário seleciona os parâmetros de números de contrações realizadas (repetições do exercício), e duração do tempo de intervalo entre as séries, com adição da duração de cada contração, isto é, o tempo em que o esforço deve ser mantido para que uma repetição seja contabilizada.
\item Modo potência - Semelhante ao modo dinâmico, estes são exercícios cíclicos, porém com a maior velocidade possível. Os parâmetros para esse exercício são os mesmos do modo dinâmico, porém aqui a velocidade do movimento faz parte dos parâmetros de controle.
\end{enumerate}

\section{Desenvolvimento}

O e-lastic começou como um produto desenvolvido para interfacear com usuário por modo de um dispositivo específico conectado ao sensor de força via cabo. A seleção dos exercícios e o próprio feedback para o usuário eram realizados por meio desse dispositivo, fazendo interação com o usuário por meio de um display LCD e um potenciômetro para seleção de parâmetros de exercício, e o feedback sendo feito por meio de LEDs e sinais sonoros, bem como o próprio display LCD. Esse produto é resultado do trabalho de mestrado em processamento de sinais da aluna Fernanda Sampaio Teles, que resultou num pedido de patente com número de registro BR 5120130007631.

Por volta de 1 vez ao mês, foram realizadas reuniões com Fernanda e também Gabriela Sartório, também aluna de processamento de sinais trabalhando nesse projeto. Daqui em diante serão referenciadas neste documento como \textit{product owners}.

Os requisitos para o aplicativo foram a principio retirados do próprio comportamento dos primeiros protótipos. Esses comportamentos foram descritos e detalhados pelos product owners, que não só descreveram como os protótipos funcionavam, como também foi discutido como o aplicativo deveria se comportar de forma diferente dos mesmos, com adições de novas funcionalidades para o aplicativo sendo desenvolvido.

A ideia inicial do aplicativo, e foco principal do desenvolvimento da equipe ao longo desse semestre, foi fazer com que o aplicativo sendo desenvolvido para plataforma Android tenha no mínimo as mesmas funcionalidades dos primeiros protótipos. 

O desenvolvimento do e-lastic app ao longo do semestre foi planejado e executado juntamente com a disciplina de Manutenção e Evolução de Software (MES). Desde o início do semestre, foi preparada uma equipe na disciplina para o desenvolvimento do aplicativo desde o início, de forma a preparar a base da arquitetura. Essa base será utilizada na segunda fase deste trabalho, quando o desenvolvimento será focado em alcançar todos os objetivos traçados nas reuniões com os product owners, que se resumem basicamente na construção e entrega do aplicativo completo e funcional.

Para este trabalho em conjunto com a disciplina de MES durante o semestre, o objetivo principal é conseguir uma base de arquitetura e alcançar dois tipos de exercícios com funcionalidade implementada: exercício dinâmico e isométrico. A interface gráfica foi deixada em segundo plano durante esse período de desenvolvimento, com o foco voltado então em uma base estável e manutenível com as funcionalidades básicas implementadas e com seus respectivos testes.

\subsection{Ciclo de desenvolvimento}		

Durante o desenvolvimento, foram utilizadas práticas ágeis retiradas do Scrum. Embora o Scrum não esteja no escopo deste trabalho, informações sobre o mesmo podem ser obtidas em \citeonline{scrum}.

Devido ao ritmo de trabalho mais lento proporcionado pelo tempo curto da disciplina de MES, as sprints foram planejadas para durar 15 dias (2 semanas). Inicialmente foram criadas várias histórias de usuário (user stories) e montado um backlog considerando a reunião inicial com os product owners. Novas histórias surgiam, bem como histórias iam sendo atualizadas conforme os requisitos ficavam mais claros com as reuniões subsequentes. A cada 2 semanas então eram consideradas as histórias de usuário do backlog e priorizadas para a iniciar a sprint. Nas medidas utilizadas pela equipe, com pontuação de 1 a 5 para cada história utilizando técnica do planning poker, foi implementada uma média de 7 story points por sprint. É importante ressaltar que histórias não completadas não entram nem parcialmente nesse cálculo e são jogadas para as sprints seguintes, e portanto para uma equipe inexperiente em constante aprendizado é possível inferir que mais esforço foi dedicado do que o próprio velocity pode indicar, representando a curva de aprendizado da equipe. Muito do meu esforço foi direcionado ao aprendizado da equipe para alavancar o desenvolvimento do aplicativo.

Todo trabalho realizado foi feito utilizando técnica de programação em pares. Um quadro mapeando os integrantes que já trabalharam juntos esteve em constante atualização para que a equipe circulasse da melhor forma possível. Participei do desenvolvimento dessa forma várias vezes ao longo do semestre, mesmo esse não sendo o objetivo principal da minha contribuição no desenvolvimento, com o intuito de compartilhar de parte do conhecimento da plataforma que obtive no projeto com parceria UnB-Positivo de desenvolvimento de aplicativos Android, bem como conhecimento que adquiri em estágio no ICRI-SC (Intel Collaborative Research Institute for Security Computing), localizado em Darmstadt, Alemanha.

O código sendo desenvolvido se encontra disponível em um repositório aberto\footnote{\url{http://gitlab.com/biodyn/biodynapp}} para qualquer desenvolvedor através da plataforma gitlab.com, e a equipe de alunos de MES não tem acesso direto de escrita para esse repositório. Para contribuir para o projeto, a equipe trabalha em uma cópia do repositório original (fork). Quando alcança algo significativo para o projeto, a equipe então cria um \textit{merge request} pela própria plataforma gitlab, que é então avaliado e aceito por mim, ou rejeitado e mudanças são solicitadas. O mesmo procedimento pode ser realizado por qualquer desenvolvedor que deseja contribuir com o projeto. Como já dito, o software sendo desenvolvido é livre e portanto tem código aberto disponibilizado online. A única parte do produto e-lastic que será comercializada é o componente físico, com o módulo central de processamento, sensor de força e implemento elástico.

\subsection{Equipe}
Composta por 5 alunos da disciplina de MES, a equipe contém apenas graduandos em Engenharia de Software da Universidade de Brasília campus UnB Gama. Todos eles tem um tempo a dedicar ao desenvolvimento correspondente ao tempo de uma disciplina de 4 créditos, com 4 horas-aula e mais 4 horas semanais adicionais por integrante, totalizando 8 horas por semana de dedicação para cada um dos integrantes da equipe.

Todos os integrantes não tinham a priori nenhuma experiência no desenvolvimento de aplicativos para a plataforma Android, porém todos eles tinham algum conhecimento em linguagem de programação Java, necessária ao desenvolvimento, bem como experiência em atividades de verificação e validação, banco de dados, entre outras áreas. Todo esse conhecimento foi adquirido ao longo do curso de graduação em Engenharia de Software que está em andamento, provendo o background necessário para que eles pudessem trabalhar neste projeto durante o semestre. A equipe também apresentou conhecimento suficiente em gerência de configuração e controle de versionamento, o que ajudou significativamente em relação a manutenção do repositório do projeto.

Um dos integrantes da equipe faz o papel de \textit{coach}, coordenando o restante da equipe para alcançar os objetivos de cada sprint e tomando frente da organização da equipe ao longo dos ciclos de desenvolvimento. Embora tenha essa responsabilidade adicional, esse integrante participava das atividades de desenvolvimento tanto quanto os demais, e foi selecionado para esse papel pelo professor da disciplina por ter mais experiência que os demais em metodologia ágil de desenvolvimento de software.

O principal papel que exerci ao longo do desenvolvimento foi em relação ao estabelecimento da arquitetura do aplicativo sendo desenvolvido, visto que a equipe não tinha experiência com componentes do Android e não conhecia a própria estrutura de um aplicativo para essa plataforma. Dessa forma, fiquei responsável por criar a estrutura base para o desenvolvimento do aplicativo, bem como revisar todos os códigos submetidos para o repositório principal, e solicitar modificações caso não estivessem de acordo com a arquitetura base do repositório principal, ou não estivesse com qualidade aceitável. Como participei ativamente do desenvolvimento, não houve muita surpresa nos merge requests e portanto poucos foram rejeitados.

Durante as aulas da disciplina de MES, toda a equipe esteve presente comigo, onde era aproveitado o tempo em conjunto para a disseminação do conhecimento sobre a plataforma. Por várias vezes foram feitos DOJOs de treinamento comigo para que o conhecimento circulasse dentro da equipe. Com o mesmo objetivo, a equipe sempre trabalhou em revezamento de pares e de tarefas. Uma dupla que trabalhou com um módulo específico durante uma semana passava a trabalhar em outro na semana seguinte, ou mesmo em atividades de teste, com outra dupla assumindo o trabalho da anterior. Os demais horários da equipe variam de acordo com a semana e com a dupla que trabalhará na semana. As duplas sempre alternavam, e os horários para cada dupla dependem da disponibilidade dos integrantes. Nesses horários de desenvolvimento, não participei de forma presencial e não houve participação de toda a equipe, cada dupla trabalhava em uma parte separada da aplicação.

Como os conhecimentos em desenvolvimento Android foram adquiridos ao longo do desenvolvimento, a primeira sprint teve menor velocity que as demais, como demonstrado na imagem BLABLA.
%TODO verificar velocity da equipe e fazer grafico e referenciar na linha anterior

Embora meu papel principal fosse como arquiteto e revisor de código, também participei da codificação do aplicativo, codificando e construindo boa parte dos principais componentes desenvolvidos ao longo do semestre. Um dos motivos para isso foi dar o impulso inicial que a equipe inexperiente na plataforma precisava para trabalhar de forma mais eficiente.

\section{Estado do desenvolvimento}
%arquitetura atual do app. Componentes e sua comunicaçao indireta (possibilidade de modulo wifi no futuro em vez do bluettooth). Comparaçao da arquitetura ao MVC. Testes e suas dificuldades. gitlab, codigo aberto a forks e pull requests de qualquer um.
	%TODO falar do planejamento ate o fim do semestre