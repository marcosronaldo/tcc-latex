\chapter{e-lastic}
O E-lastic é um sistema eletrônico que monitora e controla a execução de exercícios físicos realizados com equipamento que impõe sobrecarga à movimentação de segmentos corporais por meio de resistência elástica. Geralmente o controle de sobrecarga gerada por elementos elásticos é baseado na percepção subjetiva do esforço, com base na sensação de fadiga experimentada durante o exercício, e portanto o praticante não têm controle do esforço aplicado no exercício.

O produto em desenvolvimento apresenta um aparelho portátil, voltado para o controle de atividades físicas em ambientes fechados ou abertos. Trata-se de um sistema eletrônico embarcado que realiza o processamento digital do sinal originado num sensor de força e associa essas informações com variáveis de espaço e tempo, de forma a gerar informações suficientes para o controle e prescrição de exercícios resistidos. Esse sistema eletrônico permite a acoplagem do implemento elástico para a realização do exercício a ser monitorado, e interfaceia com o usuário por meio de um aplicativo desenvolvido para um dispositivo móvel. De forma geral, durante o exercício físico, a força aplicada pelo usuário ao elemento elástico é calculada no microcontrolador e enviada juntamente com as demais informações via Bluetooth para um dispositivo móvel com o e-lastic app, que contém opções de controle para a realização do exercício físico.



 


