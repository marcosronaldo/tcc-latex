\chapter{Introdução}
\section{Contexto}
A tecnologia tem avançado muito desde o século XX em diversas áreas, e esse avanço continua bastante visível no século XXI no que diz respeito à micro eletrônica. Aparelhos e maquinas que antes custavam fortunas e ocupavam salas inteiras puderam avançar ao ponto de caber em uma única caixa. Engrenagens e inúmeras válvulas reduzidas a um micro chip. Várias ferramentas utilizadas no dia a dia de muitos trabalhadores reunidas em um único computador, que adquiriu muitas outras funções além de fazer cálculos.

Nos últimos anos, todas as funções de um computador vem sendo gradativamente transferidas para dispositivos que cabem na palma da mão. Atualmente podemos encontrar dispositivos móveis com poderosos processadores e alta capacidade de armazenamento de dados, além da conectividade que faz com que a utilização de cabos pareça insensata. A cada dia temos uma maior quantidade de informação concentrada de forma prática e segura, resultando em tarefas cotidianas automatizadas e controladas.

Da mesma forma que cresce a utilização de dispositivos móveis, crescem as empresas os desenvolvem. No mercado atual atuam diversas fabricantes de hardware, mas a grande maioria delas se concentra em três diferentes sistemas operacionais que se destacam no mercado: Android, iOS, e WindowsPhone. O Sistema Android controla a maior fatia do mercado de dispositivos móveis, se tornando mais atrativo para desenvolvedores de aplicativos, uma vez que um maior público pode ser alcançado. Por ser o único dos três sistemas que é livre, ele permite uma maior customização e dá mais liberdade para o desenvolvimento que os demais sistemas. Embora já seja um sistema bastante consolidado, existem diversos estudos em todo o mundo sobre como melhorá-lo em diversos aspectos, desde a interface de usuário a complexos módulos de segurança. Esses estudos resultam em contribuições frequentes da comunidade de desenvolvedores, tornando o sistema cada vez mais robusto.

O Android vem avançando em uma velocidade impressionante desde sua criação, e não apenas para celulares. Hoje em dia é possível encontrar o sistema em \textit{tablets}, televisões, relógios, e até mesmo em automóveis.

Inúmeros projetos são desenvolvidos com dispositivos móveis nos dias atuais, em especial no Android. É possível que ao ligar a televisão se veja alguma reportagem sobre algum novo aplicativo que facilita alguma tarefa importante. Diversos projetos inovadores surgem em diversas áreas e contextos com a utilização de dispositivos móveis, dentro e fora das universidades. Até mesmo monitores cardíacos podem ser substituídos por um aplicativo com conexão bluetooth e um outro dispositivo pequeno para coleta de dados, reduzindo custos.
\section{Proposta}
Equipamentos elásticos vem sendo amplamente utilizados em treinamentos físicos e fisioterápicos. Suas vantagens incluem portabilidade, maior liberdade de movimentos e menor custo. Entretanto, esses exercícios não são precisamente controlados fora de laboratórios, e sua utilização é dependente da percepção subjetiva do esforço, com base na sensação de fadiga experimentada durante o exercício, podendo causar danos musculares ao praticante.

O e-lastic é um sistema eletrônico em desenvolvimento que monitora e controla a execução de exercícios físicos realizados com implementos elásticos. Os primeiros protótipos tem um módulo físico de feedback para o usuário que apresenta as informações necessárias para o praticante do exercício físico por meio de um display, sinais sonoros e LEDs para sinalização. O protótipo mais recente obtém as informações por sensores da mesma forma que os anteriores, porém sua interface de transmissão dessas informações é a tecnlogia bluetooth.

Durante esse trabalho, será desenvolvido um aplicativo móvel que faz conexão com o dispositivo físico de coleta de informações e controlará o exercício sendo executado. Em adição aos primeiros protótipos, os dados serão apresentados de forma mais clara na tela do usuário e vários tipos de feedback para os exercícios podem ser adicionados graças aos recursos encontrados nos dispositivos móveis, como por exemplo a vibração do aparelho. O aplicativo terá um breve tutorial sobre a utilização do aparelho com o implemento elástico, e auxiliará o usuário em todas as etapas do exercício. Os parâmetros serão configurados via \textit{touchscreen} e poderão ser salvos no aplicativo para serem utilizados da próxima vez que o exercício for praticado.

Esse aplicativo será desenvolvido para a plataforma Android utilizando várias técnicas da Engenharia de Software. Todos os processos de coleta e análise de requisitos, design de arquitetura, construção, verificação e validação, e manutenção, serão trabalhados para que se alcance um produto funcional e de qualidade, com a maior eficiência possível. É importante ressaltar que o software desenvolvido será gratuito, totalmente livre e aberto a contribuições de desenvolvedores.
\section{Objetivos}
O objetivo geral deste trabalho é o desenvolvimento e disponibilização do aplicativo em funcionamento na plataforma Android para que seja utilizado em conjunto com o hardware e-lastic, que tem expectativa de comercialização num futuro não muito distante.

objetivos específicos:
\begin{itemize}
\item Estudo da plataforma Android e seus componentes
\item Aplicação de conceitos da Engenharia de Software na construção de um aplicativo Android
\item Entrega e distribuição do aplicativo e-lastic na \textit{Play Store}
\end{itemize}

O trabalho está dividido em 2 etapas, sendo a primeira etapa trabalhada neste documento. O objetivo da primeira etapa deste trabalho foi focado no estudo da plataforma e na construção da base arquitetural para o desenvolvimento do aplicativo completo na etapa seguinte, que consistirá no término do desenvolvimento, validação dos resultados, e distribuição do aplicativo.
\section{Estrutura do trabalho}
Este documento está dividido em 3 capítulos. No capítulo ~\ref{cap:android-os}, é apresentada uma descrição geral da plataforma Android. O capítulo ~\ref{cap:eswandroid} trata da aplicação da engenharia de software no desenvolvimento de um aplicativo para a plataforma Android, apresentando formas de aplicar conceitos utilizando o ambiente de desenvolvimento da plataforma e sua API. O capítulo ~\ref{cap:elastic} apresenta mais informações sobre o estudo de caso e alguns resultados e detalhes acerca da primeira etapa de desenvolvimento. Por fim, são apresentadas as considerações finais sobre a utilização da plataforma Android e discussões sobre a primeira etapa do desenvolvimento.