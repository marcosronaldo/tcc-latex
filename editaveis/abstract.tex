\begin{resumo}[Abstract]
  
In recent years, the use of smart mobile devices have increased in everyday life, to the point of replacing computers for the most basic functions. It has emerged innovative projects that leverage the constant presence of these devices by modifying the way we see and perform various daily tasks. It has also increased the use of free and open source systems, such as the Android platform which currently holds large share of the mobile device market. Occupying a key role in developing these new applications, software engineering brings engineering concepts to the development and maintenance of software products, making the development process more organized and efficient, aiming for improvement in product quality. The main objective of this study is to monitor source code metrics in the Android API, essentially object-oriented metrics, and study the evolution of its values in the different versions of the API, to verify the similarities with system applications, and then check the possibility of using the data obtained to assist in application development.
   \vspace{\onelineskip}
 
  \noindent 
  \textbf{Key-words}: Software Engineering. Source code metrics. Android.
\end{resumo}
