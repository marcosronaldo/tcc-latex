\begin{resumo}

Nos últimos anos, a utilização de dispositivos móveis inteligentes se mostra cada vez mais presente no dia a dia das pessoas, ao ponto de substituir computadores na maioria das funções básicas. Com isso, vem surgindo projetos cada vez mais inovadores que aproveitam da presença constante desses dispositivos, modificando a forma que vemos e executamos várias tarefas cotidianas. Vem crescendo também a utilização de sistemas livres e de código aberto, a exemplo da plataforma Android que atualmente detém extensa fatia do mercado de dispositivos móveis. Ocupando papel fundamental na criação dessas novas aplicações, a engenharia de software traz conceitos de engenharia ao desenvolvimento e manutenção de produtos de software, tornando o processo de desenvolvimento mais organizado e eficiente, sempre com o objetivo de melhorar a qualidade do produto. O objetivo geral deste trabalho é realizar uma análise da API do sistema Android a partir de uma análise estática de seu código fonte, e então avaliar a possibilidade de utilizar os resultados como referência de qualidade para o desenvolvimento de aplicativos desenvolvidos para o Android. Traçar então um perfil que caracteriza o próprio sistema a partir de suas métricas de código fonte, assim como traçar esse perfil para os aplicativos do sistema e de terceiros e verificar o grau de aproximação que eles encontram em seu design. 


\vspace{\onelineskip}
    
 \noindent
 \textbf{Palavras-chaves}: Engenharia de Software. Métricas de código. Android.

\end{resumo}
