\begin{resumo}

Nos últimos anos, a utilização de dispositivos móveis inteligentes se mostra cada vez mais presente no dia a dia das pessoas, ao ponto de substituir computadores na maioria das funções básicas. Com isso, vem surgindo projetos cada vez mais inovadores que aproveitam da presença constante desses dispositivos, modificando a forma que vemos e executamos várias tarefas cotidianas. Vem crescendo também a utilização de sistemas livres e de código aberto, a exemplo da plataforma Android que atualmente detém extensa fatia do mercado de dispositivos móveis. Ocupando papel fundamental na criação dessas novas aplicações, a engenharia de software traz conceitos de engenharia ao desenvolvimento e manutenção de produtos de software, tornando o processo de desenvolvimento mais organizado e eficiente, sempre com o objetivo de melhorar a qualidade do produto. O objetivo desse trabalho é trazer os fundamentos da engenharia de software para o desenvolvimento de um aplicativo para o sistema Android, aplicando atividades de requisito, desenho, construção, teste e manutenção, bem como o gerenciamento do desenvolvimento como um todo. Como estudo de caso, será utilizado o e-lastic, um dispositivo de controle para exercícios físicos com elástico, tendo como foco a construção de um aplicativo de código aberto para comunicação direta e recepção de dados, gerando \textit{feedback} sobre os exercícios realizados.

\vspace{\onelineskip}
    
 \noindent
 \textbf{Palavras-chaves}: Engenharia de Software. Dispositivos móveis. Android.

\end{resumo}
